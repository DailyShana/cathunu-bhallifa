\documentclass[12pt,UTF8]{ctexbook}
\usepackage{ctex}
\usepackage{array}
\usepackage{graphicx}
\usepackage{wrapfig}
\usepackage[table,dvipsnames]{xcolor}
\usepackage{tabularx}
\usepackage{amsmath}
\usepackage{amssymb}
\usepackage{xfrac}
\usepackage{eucal}
\usepackage{titlesec}
\usepackage{amsthm}
\usepackage{tikz-cd}
\usepackage{enumitem}
\usepackage{verbatim}
\usepackage{fontspec,xunicode,xltxtra}
\usepackage{xeCJK} 
\usepackage{caption}
\usepackage{thmtools, thm-restate}

\definecolor{gl}{RGB}{246, 252, 240}
\definecolor{gd}{RGB}{236, 244, 230}
\definecolor{bg}{RGB}{242, 244, 228}

\setCJKmainfont[BoldFont=STZhongsong]{STSong}
\setCJKmonofont{simkai.ttf} % for \texttt
\setCJKsansfont{simfang.ttf} % for \textsf
\setlength\parskip{8pt}
\setlength{\fboxsep}{12pt}
\renewcommand\thesection{\arabic{chapter}.\arabic{section}}
\newcommand{\arccot}{\operatorname{arccot}}
\newcommand{\dlim}[1]{^{\color{gray}\prime}#1}
\newcommand{\lian}[1]{
    \underset{#1}{\operatorname{lian}\,}
}
\newtheorem{df}{定义}[section] 
\newtheorem{pp}{命题}[section]
\newtheorem{tm}{定理}[section]
\newtheorem{ex}{例子}[section]
\newtheorem{et}{例题}[section]
\newtheorem{sk}{思考}[section]
\newtheorem*{po}{公理}
\newtheorem*{so}{解答}
\renewenvironment{proof}{\paragraph{\textbf{证明:}}}{\hfill$\square$}
\newtheorem{xt}{习题}[section]
\newtheorem{cor}{推论}[pp]

% 列举环境的行间距
\setenumerate[1]{itemsep=0pt,partopsep=0pt,parsep=0pt,topsep=0pt}
\setitemize[1]{itemsep=0pt,partopsep=0pt,parsep=0pt,topsep=0pt}
\setdescription{itemsep=0pt,partopsep=0pt,parsep=0pt,topsep=0pt}
% 章节字体大小
\titleformat{\section}{\zihao{-2}\bfseries}{ \thesection }{16pt}{}
% 封面
\title{\zihao{0} \bfseries 第三册习题集}
\author{\zihao{2} \texttt{大青花鱼}}
% \date{\bfseries\today}
\date{}
% 正文
\begin{document}
\maketitle
\tableofcontents
\newpage


\chapter{无穷}

\section{无穷集合的势}

\section{常见无穷集合的势}

\section{可数和不可数}

\chapter{连续函数的变化}

\section{函数在一点的变化}

\begin{df}{\textbf{散整式}}
    如果整式$P$的系数在数域$\mathbb{A}$中,根也都在$\mathbb{A}$中,就说$P$是(在$\mathbb{A}$上的)\textbf{散整式}。
    或者说$P$是\textbf{散}的,在$\mathbb{A}$上\textbf{散开}。散整式$P$可以写成:
    $$P(X) = c(X - x_1)^{m_1}(X - x_2)^{m_2}\cdots(X - x_k)^{m_k}$$
    的形式。其中$c\in \mathbb{A}$,$c\neq 0$,根$x_1 < x_2 < \cdots < x_k$是$\mathbb{A}$中系数。$m_0, m_1, \cdots, m_k$是正整数。
    
\end{df}

\section{微变的运算法则}

\section{常见函数的微变}

\section{微变函数}


\begin{ex}\label{ex:3-2-4-0}
    设$f$是开区间$I$上可微的函数,下面要证明微变函数$\partial f$在$I$上满足介值定理,
    即如果对$I$中两点$a,b$有$\partial f(a) < \partial f(b)$,那么存在$c\in I$
    使得$\partial f(a) < \partial f(c) < \partial f(b)$。

    \indent 1. 设$a,b\in I$,$\partial f(a) < \partial f(b)$,对$z \in (\partial f(a), \partial f(b))$,
    证明:有正实数$H$,使得只要$0 < h < H$,就有:
    $$ \frac{f(a + h) - f(a)}{h} < z < \frac{f(b + h) - f(b)}{h}. $$
    \indent 2. 证明:存在$h>0$,$y\in I$,使得$y+h \in I$,且
    $$ \frac{f(y + h) - f(y)}{h} = z.$$
    \indent 3. 证明:存在$x\in I$使得$z = \partial f(x)$。\\
    \indent 4. 证明$f$把区间$I$映射到一个区间。\\
    \indent 5. 考虑函数$x\mapsto x^2\sin{\left(\frac{1}{x^2}\right)}$。证明$f$在$[0,1]$上可微。
    $f$的微变函数是否在$[0,1]$上连续?给出集合$\partial f([0,1])$的特征。你能得出什么结论?
\end{ex}


\begin{so}
    \indent 1. 根据定义,微变率是变率的极限。因此,对任意实数$r>0$,总有$H>0$,
    使得只要$0 < h < H$,就有:
    $$\left| \frac{f(a + h) - f(a)}{h} - \partial f(a) \right| < r.$$
    对$\partial f (b)$也一样。选择$r$使得$\partial f(a) + r < z$,$\partial f(b) - r > z$。
    比如$z - \partial f(a)$和$\partial f(b) - z$中较小者的一半。
    再选择使得相应变率足够接近$\partial f (b)$、$\partial f (b)$的$H$。
    这样,只要$0 < h < H$,就有:
    \begin{align}
        \left| \frac{f(a + h) - f(a)}{h} - \partial f(a) \right| < r, \notag \\
        \left| \frac{f(b + h) - f(b)}{h} - \partial f(b) \right| < r. \notag
    \end{align}
    从而就有:
    $$ \frac{f(a + h) - f(a)}{h} < z < \frac{f(b + h) - f(b)}{h}. $$
    
    \indent 2. 给定$0 < h < H$,考虑函数
    $$\phi : x\mapsto \frac{f(x + h) - f(x)}{h}. $$
    则$\phi$是$[a, b-h]$上的连续函数。因此,由于$z\in (\phi(a), \phi(b))$,根据介值定理,
    有$y\in (a, b-h)$,使得
    $$ \frac{f(y + h) - f(y)}{h} = z.$$

    \indent 3. 根据微分零值定理,存在$x\in (y, y+h)$使得$\partial f(x) = \frac{f(y + h) - f(y)}{h} = z$。

    \indent 4. 考虑像集$\partial f(I)$,我们在前几问中证明了:
    在$\partial f(I)$任取两个元素$\partial f(a) < \partial f(b)$,
    则开区间$(\partial f(a), \partial f(b))$中任一点都在$\partial f(I)$中。
    这说明$\partial f(I)$是一个区间。

    \indent 5. 容易证明$f$在$(0, 1]$上可微,微变函数为:
    $$ \partial f(x) = 2x\sin{\left(\frac{1}{x^2}\right)} - \frac{2}{x} \cos{\left(\frac{1}{x^2}\right)}$$
    对于$0$点,考虑变率:
    \begin{align}
        \left| \frac{f(h) - f(0)}{h} \right| \leqslant |h|\left|\sin{\left(\frac{1}{h^2}\right)}\right| \leqslant |h| \notag
    \end{align}
    因此$h$趋于$0$时,变率趋于$0$。这说明$f$在$0$处可微,微变率为$0$。

    另外,对任意正整数$n$,取$x = \frac{1}{\sqrt{2n \pi}}$,则$\frac{1}{x^2} = 2n \pi$。它的正弦值为$0$,余弦值为$1$。
    于是$\partial f(x) = -2\sqrt{2n\pi}$。这说明$x$按数列$\{\frac{1}{\sqrt{2n \pi}}\}$趋于$0$时,
    $\partial f$的值趋于负无穷大。因此$\partial f$在$0$处不连续。
    当然,我们也可以取使得$\frac{1}{x^2}$的正弦值为$0$,余弦值为$-1$的$x$,类似可以让$\partial f$的值趋于正无穷大。
    使用前几问的结果可知:$\partial f([0,1])$是全体实数集。这也从另一方面印证$\partial f$不连续。
\end{so}

\begin{ex}\label{ex:3-2-4-1}
    设$f$在闭区间$[a, b]$上连续,在开区间$(a, b)$上可微。$f(a) = f(b) = 0$。
    $c$是数轴上不属于$[a, b]$的一点。
    证明:存在$(a, b)$上一点,函数$f$在该点的切线与横坐标轴交于$(c, 0)$点。
\end{ex}

\begin{proof}
    对$(a, b)$上的点$t$,$f$过点$t$的切线方程为:
    $$ y = f(t) + \partial f(t) (x - t).$$
    切线过$(c, 0)$点,说明以下关系成立:
    $$ \partial f(t) (t - c) - f(t) = 0$$
    考虑构造一个关于$t$的函数$g(t)$,使得$g(a) = g(b)$,且$\partial g = 0$当且仅当上式等于$0$。
    这样,我们使用微变零值定理,就能得到结论。

    直接让$\partial g = \partial f(t) (t - c) - f(t)$,发现没有简单的构造方法。
    考虑让$\partial g$为分式,$\partial f(t) (t - c) - f(t)$作为分母。这样思考下,我们构造:
    $$ g: t\mapsto \frac{f(t)}{t - c}.$$
    则$g(a) = g(b) = 0$。而求微得到:
    $$ \partial g = \frac{\partial f(t) (t - c) - f(t)}{(t - c)^2}.$$
    这就是我们要找的$g$。运用微变零值定理,存在$x\in(a, b)$,使得$\partial g = 0$,即
    $\partial f(x) (x - c) - f(x) = 0$。因此$f$在$x$的切线过$(c, 0)$点。
\end{proof}

\textbf{注意}:本题结论在直观上很容易理解,即当点在$(a, b)$上运动时,
$f$过点的切线扫过整个数轴$[a, b]$以外的部分,不会有遗漏。


\begin{ex}\label{ex:3-2-4-10}
    如果$P$是$\mathbb{R}$上的散整式,证明对任意实数$t$,$\partial P + tP$也是散整式。
\end{ex}

\begin{so}
    把$P$写成$c(X - x_1)^{m_1}(X - x_2)^{m_2}\cdots(X - x_k)^{m_k}$的形式,它的次数是:
    $$n = m_1 + m_2 + \cdots +m_k$$
    每个根$x_i$都是$\partial P$的$m_i - 1$次根,那么它也是$\partial P + tP$的$m_i - 1$次根。
    计算重根的话,我们已经找到了$\partial P + tP$的$n - k$个根。

    现在继续找出其他的根。我们可以猜测,对每个$0<i<k$,区间$(x_i, x_{i+1})$中都有一个根。

    考虑函数$f:x\mapsto P(x)\mathrm{e}^{tx}$。$f$在闭区间$[x_i, x_{i+1}]$上连续,在开区间$(x_i, x_{i+1})$上可微,
    根据微变零值定理,$\partial f$在$(x_i, x_{i+1})$上有零点。而 
    $$ \partial f (x) = (\partial P(x) + tP(x))\mathrm{e}^x,$$
    $\mathrm{e}^x$总大于零,所以这个零点就是$\partial P + tP$的根。

    这样,我们又找到了$k-1$个根。合共$n-1$个根。如果$t=0$,那么$\partial P + tP$是$n-1$次多项式,
    这些根就是它所有的根。如果$t\neq 0$,那么$\partial P + tP$可以写成这$n-1$个根的散整式和一个一次式的乘积。
    于是最后一个根也是实数。这说明$\partial P + tP$总是散的。

\end{so}


\begin{ex}\label{ex:3-2-4-20}
    $f$是$\mathbb{R}^+$到$\mathbb{R}$的连续函数,且在$(0,\infty)$上可微。
    已知$f(0) = 0$,$\lian{x\to\infty} f(x) = 0$。
    证明,存在正实数$c$,使得$\partial f(c) = 0$。
\end{ex}

\begin{so}
    如果$f$恒等于$0$,那么任取正实数即可。

    如果$f$不恒等于$0$,不妨设$f$有正值$f(a) > 0$。
    由于$\lian{x\to\infty} f(x) = 0$,所以对$\frac{f(a)}{2} > 0$,存在$R > 0$,
    使得只要$x > R$,就有$f(x) < \frac{f(a)}{2}$。比如,$f(R+1) < \frac{f(a)}{2}$。

    因此,根据介值定理,区间$(0, a)$和$(a, R+1)$中,各有一点$x_1$、$x_2$,
    使得$f(x_1) = f(x_2) = \frac{f(a)}{2}$。

    因此,根据微变零值定理,存在$c\in(x_1, x_2)$,使得$\partial f(c) = 0$。

\end{so}

\begin{ex}\label{ex:3-2-4-30}  % 洛必达法则(部分)
    函数$f$、$g$在闭区间$[a, b]$上连续,在开区间$(a, b)$上可微。\\
    \indent 1. 如果$\partial g$在$(a, b)$上不等于$0$,证明$g$在$(a, b)$上不等于$g(b)$。\\
    \indent 2. 给定$t\in[a, b)$。记$v = \frac{f(t) - f(b)}{g(t) - g(b)}$。考虑定义在$[a,b]$上的函数$h$:
    $$ h: \,\, x \mapsto f(x) - v\cdot g(x)$$
    验证$h(t) = h(b)$,从而证明:
    $$ \exists c \in [t, b), \,\,\, \mbox{使得} \,\,\, \frac{f(t) - f(b)}{g(t) - g(b)} = \frac{\partial f(c)}{\partial g(c)}. $$
    \indent 3. 如果有实数$l$使得
    $$ \lian{x\to b^-} \frac{\partial f(x)}{\partial g(x)} = l, $$
    证明:
    $$ \lian{x\to b^-}  \frac{f(x) - f(b)}{g(x) - g(b)} = l, $$
\end{ex}

\begin{so}
    \indent 1. 使用反证法:如果有$t\in(a, b)$使得$g(t) = g(b)$,那么根据微变零值定理,
    存在$c\in(t, b)$使得$\partial g(c) = 0$,矛盾!

    \indent 2. 验证$h(t) = h(b)$:
    \begin{align}
        h(t) &= f(t) - \frac{f(t) - f(b)}{g(t) - g(b)} \cdot g(t) \notag \\
        &= \frac{f(t)\cdot \left(g(t) - g(b)\right) - g(t) \cdot \left(f(t) - f(b)\right)}{g(t) - g(b)} \notag \\
        &= \frac{f(t) g(t) - f(t) g(b) - g(t) f(t) + g(t) f(b)}{g(t) - g(b)} \notag \\
        &= \frac{f(b) g(t) - g(b) f(t)}{g(t) - g(b)} \notag 
    \end{align}
    \begin{align}
        h(b) &= f(b) - \frac{f(t) - f(b)}{g(t) - g(b)} \cdot g(b) \notag \\
        &= \frac{f(b)\cdot \left(g(t) - g(b)\right) - g(b) \cdot \left(f(t) - f(b)\right)}{g(t) - g(b)} \notag \\
        &= \frac{f(b) g(t) - f(b) g(b) - g(b) f(t) + g(b) f(b)}{g(t) - g(b)} \notag \\
        &= \frac{f(b) g(t) - g(b) f(t)}{g(t) - g(b)} \notag 
    \end{align}
    所以$h(t) = h(b)$。\\
    根据微变零值定理,存在$c\in(t, b)$,使得$\partial h(c) = 0$,因此:
    $$ \frac{f(t) - f(b)}{g(t) - g(b)} = \frac{\partial f(c)}{\partial g(c)}. $$
    这里我们可以用$\partial g(c)$做分母,是因为第一问的结论保证了它不等于$0$。

    \indent 3. 按照定义,对任意实数$r > 0$,存在$d > 0$,使得只要$x\in(b - d, b)$,就有
    $$ \left| \frac{\partial f(x)}{\partial g(x)} - l\right| < r.$$
    而根据第二问,存在$c\in(x, b)$,使得
    $$ \frac{f(x) - f(b)}{g(x) - g(b)} = \frac{\partial f(c)}{\partial g(c)}. $$
    $c\in(x, b)$,故$c\in(b - d, b)$,于是
    $$ \left| \frac{f(x) - f(b)}{g(x) - g(b)} - l\right| = \left| \frac{\partial f(c)}{\partial g(c)} - l\right| < r.$$
    这说明只要取$x\in(b - d, b)$,就有
    $$ \left| \frac{f(x) - f(b)}{g(x) - g(b)} - l\right| < r.$$
    也就是说,
    $$ \lian{x\to b^-}  \frac{f(x) - f(b)}{g(x) - g(b)} = l. $$

\end{so}

\section{多次微变}

\chapter{研究函数}
\section{增减与极值}
% 一阶微变
\section{凹凸性质}
% 凸函数,琴声不等式,拐点
\section{局部性质}
% 高阶无穷小、泰勒公式
\section{曲线的性质}
% 参数方程曲线的局部性质研究

\chapter{平直空间}
%将平面和立体空间扩展为一般的平直空间
\section{平直空间的基本性质}
\section{子空间与和空间}
\section{生成空间}
\section{基底和维数}

给定$f:\mathbb{R} \rightarrow \mathbb{R}$,$\partial^2 f + f = 0$。
如果$f(0) = \partial f(0) = 0$,那么$f = 0$。

$$ f(x) = f(0) + \partial f(0) x + \frac{\partial^2 f (c)}{2}x^2 = -\frac{f (c)}{2}x^2. \,\,\, 0 < c < x. $$

证明:反设$f(x_0) > 0$。

\chapter{连续函数的和}
\section{函数图像的面积}
\section{函数的定合}
\section{合函数}

\chapter{级数}
\section{正项级数}
\section{收敛与发散}
\section{函数的级数}


\end{document}
