\documentclass[12pt,UTF8]{ctexbook}
\usepackage{ctex}
\usepackage{caption}
\usepackage{graphicx}
\usepackage{float}
\usepackage{wrapfig}
\usepackage{array}
\usepackage[table, dvipsnames, svgnames, x11names]{xcolor}
\usepackage{colortbl}% 
\usepackage{tabularx}
\usepackage{amsmath}
\usepackage{amssymb}
\usepackage{xfrac}
\usepackage{eucal}
\usepackage{titlesec}
\usepackage{amsthm}
\usepackage{tikz-cd}
\usepackage{enumitem}
\usepackage{verbatim}
\usepackage{fontspec,xunicode,xltxtra}
\usepackage{xeCJK} 

\definecolor{gl}{RGB}{246, 252, 240}
\definecolor{gd}{RGB}{236, 244, 230}
\definecolor{bg}{RGB}{242, 244, 228}


\setCJKmainfont[BoldFont=STZhongsong]{STSong}
\setCJKmonofont{simkai.ttf} % for \texttt
\setCJKsansfont{simfang.ttf} % for \textsf
\setlength\parskip{8pt}
\setlength{\fboxsep}{12pt}
\renewcommand\thesection{\arabic{chapter}.\arabic{section}}
\newtheorem{df}{定义}[section] 
\newtheorem{pp}{命题}[section]
\newtheorem{tm}{定理}[section]
\newtheorem{ex}{例子}[section]
\newtheorem{sk}{思考}[section]
\newtheorem{po}{公理}
\newtheorem*{so}{解答}
\newenvironment{proof2}{\paragraph{\textbf{证明:}}}{\hfill$\square$}
\newtheorem{xt}{习题}[section]
\newtheorem{cor}{推论}[pp]
\renewcommand{\proofname}{\indent\bf 证明}
\renewcommand{\qedsymbol}{\hfill$\square$}
% 列举环境的行间距
\setenumerate[1]{itemsep=0pt,partopsep=0pt,parsep=0pt,topsep=0pt}
\setitemize[1]{itemsep=0pt,partopsep=0pt,parsep=0pt,topsep=0pt}
\setdescription{itemsep=0pt,partopsep=0pt,parsep=0pt,topsep=0pt}
\setlength{\intextsep}{2pt}%
\setlength{\columnsep}{2pt}%
% 新函数
\renewcommand\parallel{\mathrel{/\mskip-4mu/}}
% 章节字体大小
\titleformat{\section}{\zihao{-2}\bfseries}{ \thesection }{16pt}{}
% 封面
\title{\zihao{0} \bfseries 第六册}
\author{\zihao{2} \texttt{大青花鱼}}
% \date{\bfseries\today}
\date{}
% 正文
\begin{document}
\maketitle
\tableofcontents
\newpage

\chapter{向量}
第五册中,我们学习了用三角函数解三角形。三角函数是定量研究平面形的利器。
不过,三角函数本身并不是简单的函数。我们目前只能通过查表的方式得到函数值。
这让我们思考,能不能打造一种更方便定量研究的体系呢?

回顾我们对平面形的研究,我们从几条公理出发,得出点、直线、三角形、圆等形状之间的定性关系。
公理体系的缺陷在于没有与数紧密结合。比如,“两点之间直线最短”,除了定性的“最短”,没有提供别的信息。
我们需要一种根本上和数量结合的体系,来理解各种平面形状。

此外,公理体系中并没有强调运动的概念。我们说点运动形成了线,旋转形成角度和圆,
但并没有相关的工具来描述具体的运动。我们需要一种根本上和运动结合的体系,来理解形状之间的关系。

\section{点、向量和直线}
学习有理数的时候,我们使用数轴上的点表示。每个点代表一个实数。两点重合,
当且仅当它们代表同一个数。这种表示方法把数和直线上的点牢牢绑在一起。
我们可以用数的关系表示直线上点的关系。数轴使我们可以定量理解直线。

至于平面中的点,我们用相互垂直的数轴定义了点的坐标。每个点代表一个有序数对。两个数按顺序排列,对应平面中一点。

能不能像数轴一样,用一个量代表平面中一点呢?数轴之所以能用一个数代表一个点,
是因为直线只有两个方向,使用正负号就可以代表方向。平面中不止两个方向,我们无法用正负来表示方向了。
为此,我们引入一个新的量来代表平面中的点:\textbf{向量}。

自然数、有理数、实数都有自己的运算法则。向量作为代表点的量,需要满足怎样的运算法则呢?
我们从运动出发,给出以下的法则:
\begin{enumerate}
    \item 向量的加法就是平移:两个向量相加得到另一个向量。向量的加法满足结合律和交换律。
    \item 零向量表示静止不变:存在这样一个向量,任何向量与它相加,仍然是自己。
    这个向量叫做\textsl{零向量}。零向量不定义方向,也可以说它与任何向量同向或反向。它对应的点称为\textbf{原点}。
    \item 从每个非零向量,引出一根数轴:任何实数乘以向量,得到方向相同或相反的向量。
    这个运算称为\textbf{数乘运算}。数乘运算对应图形的放缩。
    \item 放缩和四则运算相容:数轴上可以做数的运算。
    \item 平移和放缩相容:先平移再放缩,和先放缩再平移,结果一样。
\end{enumerate}
让我们用数学语言把这些法则更具体地写出来。我们把平面看作集合,记为$\mathbb{V}$,其中的元素称为向量或点,
用粗体字母表示,以便和代表数的量区分:
\begin{enumerate}
    \item 加法结合律:$\forall \,\, \mathbf{a}, \mathbf{b}, \mathbf{c} \in \mathbb{V}$,$\mathbf{a}+ (\mathbf{b} + \mathbf{c}) = (\mathbf{a} + \mathbf{b}) + \mathbf{c}$。
    \item 加法交换律:$\forall \,\, \mathbf{a}, \mathbf{b} \in \mathbb{V}$,$\mathbf{a} + \mathbf{b} = \mathbf{b} + \mathbf{a}$。
    \item 存在零向量:$\forall \,\, \mathbf{a} \in \mathbb{V}$,$\mathbf{a} + \mathbf{0} = \mathbf{a}$。
    \item 放缩和四则运算相容:$\forall \,\, \mathbf{a} \in \mathbb{V}$,$1\cdot \mathbf{a} = \mathbf{a}$。$\forall s, t \in \mathbb{R}$,$(s + t)\cdot\mathbf{a} = (s\cdot\mathbf{a}) + (t\cdot\mathbf{a})$,$(s \cdot t)\cdot \mathbf{a} = s \cdot (t\cdot \mathbf{a})$。
    \item 放缩和平移相容:$\forall \,\, \mathbf{a}, \mathbf{b} \in \mathbb{V}$,$\forall \,\, t \in \mathbb{R}$,$t\cdot(\mathbf{a} + \mathbf{b}) = t\cdot\mathbf{a} + t\cdot\mathbf{b}$。
\end{enumerate}
从以上法则出发,我们可以定义直线:

\begin{df}
    过原点的直线是非零向量放缩得到的集合。不过原点的直线是过原点的直线按一点平移得到的集合。
\end{df}
给定非零向量$\mathbf{a}$,$ \{t\mathbf{a} \, | \, t\in\mathbb{R}\}$是一条过原点的直线。
给定向量$\mathbf{b}$,$ \{t\mathbf{a}+\mathbf{b} \, | \, t\in\mathbb{R}\}$是一条过$\mathbf{b}$点的直线。
要注意的是,这样定义的直线是一条数轴,自然带有正方向和单位长度。

类比可以定义线段和射线:给定非零向量$\mathbf{a}$和向量$\mathbf{b}$,
$ \{t\mathbf{a}+\mathbf{b} \, | \, t\in [0, 1]\}$是端点为$\mathbf{b}, \mathbf{a} + \mathbf{b}$的线段,
$ \{t\mathbf{a}+\mathbf{b} \, | \, t \geqslant 0 \}$是以$\mathbf{b}$为端点,以$\mathbf{a}$为方向的射线。

给定非零向量$\mathbf{a}$,如果向量$\mathbf{b}$可以通过$\mathbf{a}$放缩得到,
或者说$\mathbf{b}\in \{t\mathbf{a} \, | \, t\in\mathbb{R}\}$,就称两者\textbf{共线}。
共线的向量,通过数轴,可以方便地讨论相互的位置关系。不共线的向量之间,
如何讨论位置关系呢?为此,我们要引入\textbf{平面的根本性质}:
\begin{enumerate}
    \item 给定任何非零向量$\mathbf{a}$,总有另一个向量$\mathbf{b}$,不在直线$ \{t\mathbf{a} \, | \, t\in\mathbb{R}\}$上。我们说两者\textbf{不共线}。
    \item 从不共线的向量$\mathbf{a}, \mathbf{b}$出发,经过放缩、平移,可以得到平面中任何向量。具体来说,
    任何向量都可以表示成$s\mathbf{a} + t\mathbf{b}$的形式,集合$\{s\mathbf{a} + t\mathbf{b} | s, t, \in\mathbb{R}\}$就是整个平面。
    这样的$\mathbf{a}, \mathbf{b}$称为平面的一组\textbf{基}或\textbf{基底}。
\end{enumerate}

举例来说,在直角坐标系中,我们选择了原点重合、互相垂直的两条数轴,以每条数轴上数$1$对应的点(记为$\mathbf{e}_x, \mathbf{e}_y$)出发,通过放缩和平移,
就得到平面所有的点。平面中任一点可以写成$x\mathbf{e}_x + y\mathbf{e}_y$,其中$x,y$就是点的坐标。
直角坐标系其实是一种用向量描述平面的方法。$\mathbf{e}_x, \mathbf{e}_y$就是一组基。

\begin{xt}\label{xt:0-0-10}
    \mbox{} \\
    \indent 1. 证明:零向量只有一个,任何向量乘$0$得到零向量。\\
    \indent 2. 证明:零向量乘任何数得到零向量。\\
    \indent 3. 证明:任何向量$\mathbf{a}$都有唯一的反向量$\mathbf{b}$,满足$\mathbf{a} + \mathbf{b} = \mathbf{0}$。\\
    \indent 4. 设$\mathbf{a}, \mathbf{b}$不共线,如果$s\mathbf{a} + t\mathbf{b} = \mathbf{0}$,证明:$s = t = 0$。\\
    直角坐标系$xOy$中,设$\mathbf{a} = 4\mathbf{e}_x + \mathbf{e}_y$,$\mathbf{b} = \mathbf{e}_x - 2\mathbf{e}_y$
    ,$\mathbf{b} = -\mathbf{e}_x + 2\mathbf{e}_y$。\\
    \indent 4. 在坐标轴上标出$\mathbf{a}$,$\mathbf{b}$和$\mathbf{c}$。\\
    \indent 5. 用$\mathbf{a}$和$\mathbf{b}$表示$\mathbf{e}_x$、$\mathbf{e}_y$和点$(3,0)$。\\
    \indent 6. 用$\mathbf{a}$和$\mathbf{b}$表示它们的中点以及它们所在的直线。\\
    \indent 7. 用$\mathbf{a}, \mathbf{b}, \mathbf{c}$表示顶点为$\mathbf{a}, \mathbf{b}, \mathbf{c}$的三角形三边和重心。
\end{xt}

\section{角度与长度}
根据平面的根本性质,任何向量都可以用两个不共线向量表示。接下来,我们仿照角度,给出两个向量之间的关系。
% 构造一个把两个向量对应到一个数的映射$f$。我们希望$f$具有这样的性质:
% \begin{enumerate}
%     \item 顺序不影响关系大小:$\forall \,\, \mathbf{a}, \mathbf{b} \in \mathbb{V}$,$f(\mathbf{a}, \mathbf{b}) = f(\mathbf{b}, \mathbf{a})$。
%     \item 零向量和任意向量关系为$0$:$\forall \,\, \mathbf{a} \in \mathbb{V}$,$f(\mathbf{a}, \mathbf{0}) = f(\mathbf{0}, \mathbf{a}) = 0$。
%     \item 非零向量与自身的关系总是正的:$\forall \,\, \mathbf{a} \in \mathbb{V}\setminus\{\mathbf{0}\}$,$f(\mathbf{a}, \mathbf{a}) > 0$。
%     \item 和向量放缩相容:$\forall \,\, \mathbf{a}, \mathbf{b} \in \mathbb{V},\,\,\, t\in\mathbb{R}$,$f(\mathbf{a}, t\mathbf{b}) = f(t\mathbf{a}, \mathbf{b}) = tf(\mathbf{a}, \mathbf{b})$。
%     \item 和向量平移相容:$\forall \,\, \mathbf{a}, \mathbf{b}, \mathbf{c} \in \mathbb{V}$,$f(\mathbf{a}, \mathbf{b}+\mathbf{c}) = f(\mathbf{a}, \mathbf{b}) + f(\mathbf{a}, \mathbf{c})$。
% \end{enumerate}
给定平面基底$\mathbf{e}_1, \mathbf{e}_2$,我们给出这样一个映射$f$:
$$ \forall \,\, s_1, s_2, t_1, t_2 \in \mathbb{R}, \quad f(s_1\mathbf{e}_1 + s_2\mathbf{e}_2, t_1\mathbf{e}_1 + t_2\mathbf{e}_2) = s_1t_1 + s_2 t_2.$$
$f$把两个向量对应到一个实数。它满足以下五个性质:
\begin{enumerate}
    \item 顺序不影响关系大小:
    \begin{align}
         & f(s_1\mathbf{e}_1 + s_2\mathbf{e}_2, t_1\mathbf{e}_1 + t_2\mathbf{e}_2) \notag \\
        =& s_1t_1 + s_2 t_2 = t_1s_1 + t_2s_2 \notag \\
        =& f(t_1\mathbf{e}_1 + t_2\mathbf{e}_2, s_1\mathbf{e}_1 + s_2\mathbf{e}_2). \notag
    \end{align}
    \item 零向量和任意向量关系为$0$:
    $$f(s_1\mathbf{e}_1 + s_2\mathbf{e}_2, \mathbf{0}) = f(s_1\mathbf{e}_1 + s_2\mathbf{e}_2, 0\mathbf{e}_1 + 0\mathbf{e}_2) = s_1\cdot 0 + s_2\cdot 0 = 0.$$
    \item 非零向量与自身的关系总是正的:$s_1, s_2$不全为零时,
    $$f(s_1\mathbf{e}_1 + s_2\mathbf{e}_2, s_1\mathbf{e}_1 + s_2\mathbf{e}_2) = s_1^2 + s_2^2  > 0.$$
    \item 和向量放缩相容:
    \begin{align}
        & f(s_1\mathbf{e}_1 + s_2\mathbf{e}_2, t(t_1\mathbf{e}_1 + t_2\mathbf{e}_2)) \notag \\
        =& s_1tt_1 + s_2 tt_2 = t(s_1t_1 + s_2t_2) \notag \\
        =& tf(s_1\mathbf{e}_1 + s_2\mathbf{e}_2, t_1\mathbf{e}_1 + t_2\mathbf{e}_2). \notag 
    \end{align}
    \item 和向量平移相容:
    \begin{align}
         & f(s_1\mathbf{e}_1 + s_2\mathbf{e}_2, (t_1\mathbf{e}_1 + t_2\mathbf{e}_2) + (r_1\mathbf{e}_1 + r_2\mathbf{e}_2)) \notag \\
        &= s_1(t_1 + r_1) + s_2 (t_2 + r_2) = (s_1t_1 + s_2t_2) + (s_1r_1 + s_2r_2) \notag \\
        &= f(s_1\mathbf{e}_1 + s_2\mathbf{e}_2, t_1\mathbf{e}_1 + t_2\mathbf{e}_2) + f(s_1\mathbf{e}_1 + s_2\mathbf{e}_2, r_1\mathbf{e}_1 + r_2\mathbf{e}_2). \notag
    \end{align}     
\end{enumerate}
满足以上五个条件的映射$f$称为平面向量的\textbf{内积}。从第四个性质可知,向量与自身的内积总是正数。
我们把这个数的平方根叫做向量的长度,记为:
$$ \forall \,\, \mathbf{a} \in \mathbb{V}, \quad \| \mathbf{a}\| = \sqrt{f(\mathbf{a}, \mathbf{a})}. $$
两个向量之差的长度,称为向量之间的距离。
$$ \forall \,\, \mathbf{a}, \mathbf{b} \in \mathbb{V}, \quad \| \mathbf{a} - \mathbf{b}\| = \sqrt{f(\mathbf{a} - \mathbf{b}, \mathbf{a} - \mathbf{b})}. $$
如果基底$\mathbf{e}_1, \mathbf{e}_2$是直角坐标系的基,那么
\begin{align}
    \forall \,\, \mathbf{a} = x_A\mathbf{e}_x &+ y_A\mathbf{e}_y , \notag \\
    \| \mathbf{a}\| &= \sqrt{x_A^2 + y_A^2}, \notag \\
    \forall \,\, \mathbf{a} = x_A\mathbf{e}_x &+ y_A\mathbf{e}_y,\,\,\, \mathbf{b} = x_B\mathbf{e}_x + y_B\mathbf{e}_y , \notag \\
    \| \mathbf{a} - \mathbf{b}\| &= \sqrt{(x_A - x_B)^2 + (y_A - y_B)^2}. \notag
\end{align}
如果记$A, B$为向量$\mathbf{a}, \mathbf{b}$对应的点,$\| \mathbf{a} \|$就是点$A$到原点的距离$|OA|$,
$\|\mathbf{a} - \mathbf{b}\|$就是$A,B$两点之间的距离$|AB|$,也就是线段$AB$的长度。
也就是说,我们这样定义的$f$,分别与直观经验中长度和距离的概念相符合。

那么,$f$本身有什么含义呢?我们来计算$ \frac{|OA|^2 + |OB|^2 - |AB|^2}{2}.$
\begin{align}
    \frac{|OA|^2 + |OB|^2 - |AB|^2}{2} &= \frac{x_A^2 + y_A^2 + x_B^2 + y_B^2 - (x_A - x_B)^2 - (y_A - y_B)^2}{2} \notag \\
    &= x_Ax_B + y_Ay_B = f(\mathbf{a}, \mathbf{b}). \notag 
\end{align}
另一方面,余弦定理告诉我们,$ \frac{|OA|^2 + |OB|^2 - |AB|^2}{2} = |OA||OB|\cos \angle AOB$。
也就是说,$f(\mathbf{a}, \mathbf{b}) = \|\mathbf{a}\| \|\mathbf{b}\| \cos \angle AOB$。% \langle \mathbf{a}, \mathbf{b}\rangle 
内积$f$的本质是向量夹角的余弦与向量长度的乘积。通过内积,我们把角度和长度统一起来了。

向量夹角的余弦值总在$-1$和$1$之间,所以向量的内积的绝对值不大于向量长度的乘积:
$$ |x_Ax_B + y_Ay_B| \leqslant \sqrt{x_A^2 + y_A^2} \sqrt{x_A^2 + y_A^2}.$$
可以验证这个关系对任意$x_A, y_A, x_B, y_B$成立。从这个关系出发,可以得到:
$$ |AB| = \sqrt{(x_A - x_B)^2 + (y_A - y_B)^2} \leqslant \sqrt{x_A^2 + y_A^2} + \sqrt{x_A^2 + y_A^2} = |OA| + |OB|.$$
可以直观理解为“三角形两边之和大于第三边”或“两点之间线段距离最短”。

内积为$0$,就表示向量夹角的余弦为$0$,这时,我们说两个向量垂直。
比如令$\mathbf{a} = 2\mathbf{e}_x - \mathbf{e}_y$,$\mathbf{b} = \mathbf{e}_x + 2\mathbf{e}_y$,
那么$f(\mathbf{a}, \mathbf{b}) = 2\cdot 1 - 1\cdot 2 = 0$。在平面上画出对应的点$A,B$,
可以验证$\angle AOB = 90^\circ$。

再来看另一个映射$f_2$:
$$ \forall \,\, s_1, s_2, t_1, t_2 \in \mathbb{R}, \quad f(s_1\mathbf{e}_1 + s_2\mathbf{e}_2, t_1\mathbf{e}_1 + t_2\mathbf{e}_2) = 2s_1t_1 + s_2 t_2.$$
可以验证,$f_2$也满足$f$满足的五个性质。从$f_2$出发,我们也可以定义距离和长度:
$$ \forall \,\, \mathbf{a} = x_A\mathbf{e}_x + y_A\mathbf{e}_y, \quad \| \mathbf{a} \|_2 = f_2(\mathbf{a}, \mathbf{a}) = 2x_A^2 + y_A^2. $$
这样定义的距离和长度和我们直观经验中有些不一样,不过,我们可以验证,这样定义的距离也满足“两点之间线段最短”的性质。
$$ |2x_Ax_B + y_Ay_B| \leqslant \sqrt{2x_A^2 + y_A^2} \sqrt{2x_A^2 + y_A^2}.$$
因此,$f_2$也是内积。我们把符合直观经验的内积$f$称为\textbf{经典内积},一般称内积都默认指经典内积;把对应的长度称为向量的\textbf{模}或\textbf{范}。
$\mathbf{a}, \mathbf{b}$的经典内积记为$(\mathbf{a}\, | \, \mathbf{b})$\footnote{不至于混淆时,也常称为点积,记为$\mathbf{a} \cdot \mathbf{b}$},模记为$|\mathbf{a}|$、$|\mathbf{b}|$。

既然有余弦,自然有正弦。记$\mathbf{a}, \mathbf{b}$夹角为$\alpha$,
则$(\mathbf{a}\, | \, \mathbf{b}) = |\mathbf{a}||\mathbf{b}| \cos \alpha$,
于是,
$$ |\mathbf{a}|^2|\mathbf{b}|^2 \sin^2 \alpha = |\mathbf{a}|^2|\mathbf{b}|^2 - (\mathbf{a}\, | \, \mathbf{b})^2 $$
记$\mathbf{a} = x_A\mathbf{e}_x + y_A\mathbf{e}_y$,$\mathbf{b} = x_B\mathbf{e}_x + y_B\mathbf{e}_y$,
则
\begin{align}
    (x_A^2 + y_A^2)(x_A^2 + y_A^2) \sin^2 \alpha &= (x_A^2 + y_A^2)(x_A^2 + y_A^2) - (x_Ax_B + y_Ay_B)^2 \notag \\
    &= (x_Ay_B - x_By_A)^2 \notag \\
    |\sin \alpha| &= \frac{|x_Ay_B - x_By_A|}{\sqrt{x_A^2 + y_A^2} \sqrt{x_A^2 + y_A^2}} \notag
\end{align}
我们得出了夹角正弦的绝对值。


\chapter{三段论(下)}
\section{三段论的规则}
\section{三段论的应用}

\chapter{投影和视图}
\section{平面和立体}
\section{三视图}
\section{表面的展开}

\chapter{同余}
\begin{ex}\label{ex:3-0-0}
    $7^{65}$的个位数是多少?
\end{ex}
\begin{so}
    从$7^0,7^1,7^2,7^3\cdots$开始找规律。$7^0=1$,$7^1=7$,$7^2=49$,$7^3=343$,$7^4=2401$,$7^5=16807$。
    $7^4$和$7^0$的个位数都是$1$,$7^5$和$7^1$的个位数都是$7$。我们可以总结出这样的规律:个位数是$1$的,乘以$7$得到$7$;
    个位数是$7$的,乘以$7$得到$9$;个位数是$9$的,乘以$7$得到$3$;个位数是$3$的,乘以$7$得到$1$。

    也就是说,如果把$7^0,7^1,7^2,7^3\cdots$的个位数写成一列,应该是这个样子的:
    $$ 1, 7, 9, 3, 1, 7, 9, 3, 1, 7, \cdots$$
    用归纳法不难证明,这列数字以$4$为周期不断重复。所以,要求$7^{65}$的个位数,可以看$65$在相关的周期里处于哪个位置。
    换句话说,只要看$65$除以$4$的余数。$65 = 16 \times 4 + 1$,所以$7^{65}$的个位数和$7^1$的个位数一样,都是$7$。
\end{so}

从这个例子可以看出,两个整数除以同一个数得到相同的余数,是一个重要的性质。我们把这种性质称为\textbf{同余}。
比如,$65$和$1$除以$4$余数都是$1$,我们就说$65$和$1$模$4$同余。$7^{65}$和$7^1$除以$10$余数都是$7$,
我们说$7^{65}$和$7^1$模$10$同余,记为:
$$ 7^{65} \equiv_{10} 7^1 $$

\section{同余类}
整数除以$3$,余数有$0,1,2$三种可能。整数除以$10$,余数有$0,1,\cdots , 9$十种可能。
一般来说,给定正整数$n$,整数除以$n$,余数有$0,1,\cdots , n-1$这$n$种可能。
因此,按除以$n$的余数,可以把整数集分成$n$类。同属一类的数,模$n$同余,所以这$n$类数叫作模$n$\textbf{同余类}。
所有模$n$同余类的集合,叫作模$n$\textbf{同余系}。

每个模$n$同余类,可以写成$\{kn + a \, | \, k\in\mathbb{Z} \}$的形式。也就是说,可以看成某个数$a$不断加上或减去$n$得到的所有数的集合。这个集合是无穷的。不同的模$n$同余类,交集是空集,并集是$\mathbb{Z}$。也就是说,它们是$\mathbb{Z}$的分划。

为了方便,我们从每个模$n$同余类中选一个元素,代表这个同余类。一般来说,可以选$0,1,\cdots,n-1$个数。我们给它们加个上划线,以和作为整数的$0,1,\cdots,n-1$区分:
$$\overline{0},\overline{1},\cdots,\overline{n-1}$$

如果要强调$n$,可以把$n$加在右上角:
$$\overline{0}^n,\overline{1}^n,\cdots,\overline{n-1}^n$$

给定整数$m$,我们可以把它对应到某个模$n$同余类,称为对$n$\textbf{取模}。
比如$n=5$时,$24 \equiv_5 4$,我们把$24$对应到$\overline{4}^5$,
或者说,$24$对$5$取模,得$\overline{4}^5$。

同余关系和相等关系很像,它们是否有一样的性质呢?我们可以验证,同余关系满足以下的性质:
\begin{enumerate}
    \item $\forall \,\, a\in \mathbb{Z}, \quad a \equiv_n a$;
    \item $\forall \,\, a, b \in \mathbb{Z}$,如果$a \equiv_n b$,那么$b \equiv_n a$;
    \item $\forall \,\, a, b \in \mathbb{Z}$,如果$a \equiv_n b$,$b \equiv_n c$,那么$a \equiv_n c$。
\end{enumerate}

满足以上三个性质的二元关系(两个元素之间的关系)称为\textbf{等价关系}。数与数的等于关系是等价关系,数与数的同余关系
也是等价关系。因此,我们可以把同余关系用作同余类之间的等于关系。

整数之间有四则运算,模$n$同余类之间,也可以进行运算。以$n=5$为例子。
我们分别计算$24$和$37$除以$5$的余数,以及它们的和$61$除以$5$的余数:
$$ 24 \equiv_5 4, \,\,\, 37 \equiv_5 2 , \,\,\, 61 \equiv_5 1$$

可以发现:$ 4 + 2 \equiv_5 1$,也就是说,取模和加法可以交换顺序。
可以验证,两个同余类中各取一个元素相加,和所在的同余类,就是两者模$n$余数的和所在的同余类。
用集合的语言,可以写成:
$$\{kn + a + ln + b \, | \, k\in\mathbb{Z}, \, l\in\mathbb{Z} \} = \{kn + a + b \, | \, k\in\mathbb{Z} \}$$

所以,可以定义同余类的加法:
$$ \overline{a} + \overline{b} = \overline{a + b}$$

其中的$\overline{a + b}$指的是$a+b$所在的同余类。为了方便,我们用$a + b$作为代表。

可以验证,同余类的加法也满足结合律和交换律。这里我们只证明同余类的加法满足结合律,交换律的证明留做习题:

\begin{proof2}
    由上可知$ \overline{a} + \overline{b} = \overline{a + b}$,所以
    $$ (\overline{a} + \overline{b}) + \overline{c} = \overline{a + b}+ \overline{c} = \overline{a + b + c}.$$
    类似可得:
    $$ \overline{a} + (\overline{b} + \overline{c}) = \overline{a}+ \overline{b + c} = \overline{a + b + c}.$$
    于是
    $$ \quad \quad \quad (\overline{a} + \overline{b}) + \overline{c}  = \overline{a + b + c} = \overline{a} + (\overline{b} + \overline{c}). \quad \qedhere$$
\end{proof2}

类似可以定义同余类的减法和乘法:
$$ \overline{a} - \overline{b} = \overline{a - b}, \,\,\, \overline{a} \cdot \overline{b} = \overline{a \cdot b}$$

可以验证,同余类的减法性质和整数减法一样,同余类的乘法也满足结合律、交换律和分配律。

能否定义同余类的除法呢?我们来看一个例子。设$n=6$,考虑等式$12 \div 4 = 3$。
$12$、$4$和$3$对$6$取模,得到$0$、$4$和$3$。考虑等式$60 \div 10 = 6$。$60$、$10$和$6$对$6$取模,
得到$0$、$4$和$0$。也就是说,两个模$6$同余类中各取元素相除,商所在的同余类不是唯一的。
所以,我们没法定义模$6$同余类的除法。

再看另一个例子。设$n=5$,考虑以下的“乘法表”:
\begin{center}
    \begin{tabular}{ | p{2em}<{\centering} | p{2em}<{\centering} | p{2em}<{\centering} | p{2em}<{\centering} | p{2em}<{\centering} | p{2em}<{\centering} | }
        \hline
            $\times$   & $\overline{0}$ & $\overline{1}$ & $\overline{2}$ & $\overline{3}$ & $\overline{4}$ \\ [0.5ex] 
        \hline
        $\overline{0}$ & $\overline{0}$ & $\overline{0}$ & $\overline{0}$ & $\overline{0}$ & $\overline{0}$ \\  
        \hline
        $\overline{1}$ & $\overline{0}$ & $\overline{1}$ & $\overline{2}$ & $\overline{3}$ & $\overline{4}$ \\
        \hline
        $\overline{2}$ & $\overline{0}$ & $\overline{2}$ & $\overline{4}$ & $\overline{1}$ & $\overline{3}$ \\
        \hline
        $\overline{3}$ & $\overline{0}$ & $\overline{3}$ & $\overline{1}$ & $\overline{4}$ & $\overline{2}$ \\
        \hline 
        $\overline{4}$ & $\overline{0}$ & $\overline{4}$ & $\overline{3}$ & $\overline{2}$ & $\overline{1}$ \\
        \hline
    \end{tabular}
\end{center}

可以看出,任何模$5$同余类乘以$\overline{0}$都得到$\overline{0}$,非$\overline{0}$同余类乘以不同的同余类,结果也不同。
这说明每个同余类除以另一个同余类(非$\overline{0}$),都必然有唯一的结果。这样我们就定义了模$5$同余系里的除法。

\begin{xt}\label{xt:3-0-0}
    \mbox{}\\
    动手做一做:\\
    \indent 1. 证明同余关系满足等价关系所要求的三个性质。 \\
    \indent 2. 证明同余类的加法满足交换律。 \\
    \indent 3. 证明同余类的减法是加法的逆运算。\\
    \indent 4. 证明同余类的乘法满足结合律和交换律。\\
    \indent 5. 证明同余类的乘法满足分配律。\\
    \indent 6. 证明:如果某模$n$同余类的代表与$n$的最大公因数是$d$,则其中所有元素与$n$的最大公因数都是$d$。\\
    \indent 7. 分别画出模$3$同余系和模$4$同余系的“乘法表”。它们和模$5$同余系的“乘法表”哪些地方相同,哪些地方不同?
\end{xt}

\section{完全同余系和简化同余系}
上一节我们提到模$6$同余系无法定义除法,而模$5$同余系可以定义除法。两者有什么不同呢?
我们画出模$6$同余系的“乘法表”:
\begin{center}
    \begin{tabular}{ | p{2em}<{\centering} | p{2em}<{\centering} | p{2em}<{\centering} | p{2em}<{\centering} | p{2em}<{\centering} | p{2em}<{\centering} | p{2em}<{\centering} | }
        \hline
            $\times$   & $\overline{0}$ & $\overline{1}$ & $\overline{2}$ & $\overline{3}$ & $\overline{4}$ & $\overline{5}$ \\ [0.5ex] 
        \hline
        $\overline{0}$ & $\overline{0}$ & $\overline{0}$ & $\overline{0}$ & $\overline{0}$ & $\overline{0}$ & $\overline{0}$ \\  
        \hline
        $\overline{1}$ & $\overline{0}$ & $\overline{1}$ & $\overline{2}$ & $\overline{3}$ & $\overline{4}$ & $\overline{5}$ \\
        \hline
        $\overline{2}$ & $\overline{0}$ & $\overline{2}$ & $\overline{4}$ & $\overline{0}$ & $\overline{2}$ & $\overline{4}$ \\
        \hline
        $\overline{3}$ & $\overline{0}$ & $\overline{3}$ & $\overline{0}$ & $\overline{3}$ & $\overline{0}$ & $\overline{3}$ \\
        \hline 
        $\overline{4}$ & $\overline{0}$ & $\overline{4}$ & $\overline{2}$ & $\overline{0}$ & $\overline{4}$ & $\overline{2}$ \\
        \hline
        $\overline{5}$ & $\overline{0}$ & $\overline{5}$ & $\overline{4}$ & $\overline{3}$ & $\overline{2}$ & $\overline{1}$ \\
        \hline
    \end{tabular}
\end{center}
可以看到,这个“乘法表”和模$5$同余系的大有不同。同一行或同一列常有重复。
这说明不同的同余类乘同一个同余类得到同一个结果。比如
$$\overline{2}\times \overline{4} = \overline{5}\times \overline{4} = \overline{2}. $$
这就使我们没法定义除法。

如果我们把上面的等式稍作变化,会得到:
$$\overline{0} = (\overline{5} - \overline{2})\times \overline{4} = \overline{3} \times \overline{4}.$$
也就是说,有非$\overline{0}$的同余类相乘等于$\overline{0}$。
同余类乘法的这个性质和整数乘法完全不同。我们把这种非$\overline{0}$同余类叫做\textbf{零因子}。
整数中没有零因子:非$0$的整数相乘必然不是$0$。而只要有这种零因子存在,同余系中就会发生“不同的同余类乘同一个同余类得到同一个结果”的现象,
从而无法定义除法。

有什么办法在模$6$同余系中定义除法呢?我们可以选一部分同余类,在其中定义除法。
如果同余类$\overline{a}$的代表$a$与$6$不互素,设最大公因数是$b$,那么
$$ \frac{a}{b} \times 6 = a \times \frac{6}{b} $$
于是有$\overline{a} \times \overline{\frac{6}{b}} = \overline{0}$,出现零因子。
因此,为了避免零因子问题,我们只选和$6$互素的数所在的同余类,也就是$\overline{1}$和$\overline{5}$。
我们发现$\{\overline{1}, \overline{5}\}$中可以定义乘法和除法(但不再满足加减法)。
\begin{center}
    \begin{tabular}{ | p{2em}<{\centering} | p{2em}<{\centering} | p{2em}<{\centering} | }
        \hline
            $\times$   & $\overline{1}$ & $\overline{5}$ \\ [0.5ex] 
        \hline
        $\overline{1}$ & $\overline{1}$ & $\overline{5}$ \\
        \hline
        $\overline{5}$ & $\overline{5}$ & $\overline{1}$ \\
        \hline
    \end{tabular}
\end{center}
我们把模$6$同余系称为模$6$的\textbf{完全同余系},
把$\{\overline{1}, \overline{5}\}$称为模$6$的\textbf{简化同余系}。

一般来说,我们把模$n$同余系称为模$n$的完全同余系,在其中可以定义加减法和乘法;
把其中所有和$n$互素的同余类的集合称为模$n$的简化同余系
\footnote{通常不把$\overline{0}$计入简化剩余系,以省去讨论除以$\overline{0}$的问题。}。

\begin{tm}\label{tm:3-2-0}
    给定正整数$n$,在模$n$的简化同余系中可以定义乘法和除法。
\end{tm}
\begin{proof2}
    模$n$同余类的乘法已经定义好了。我们只需要说明:简化同余系中的同余类相乘,仍然在简化同余系中。
    这是因为与$n$互素的整数相乘,结果还是与$n$互素。\\
    接下来定义除法。除法是乘法的逆运算。比照数的除法:$a \div b = a \times \frac{1}{b}$。
    因此,只要将简化同余系中每个同余类都对应一个“倒数”,就可以用“乘以倒数”来定义除法。\\
    我们把模$n$简化同余系中的同余类用小于$n$且与$n$互素的正整数来代表,记为
    $$1 = b_1 < b_2 < \cdots < b_{\varphi(n)} = n-1.$$
    其中$\varphi(n)$是模$n$简化同余系的元素个数。考虑任一元素$b_i$,我们接下来会证明:
    $b_ib_1, b_ib_2, \cdots, b_ib_{\varphi(n)}$模$n$两两不同余。
    于是,它们中恰有一个模$n$余$1$。设$b_ib_j \equiv_n 1$,那么$b_j$就是$b_i$的“倒数”。\\
    最后用反证法证明命题:$b_ib_1, b_ib_2, \cdots, b_ib_{\varphi(n)}$模$n$两两不同余。\\
    反设命题不成立,即存在$b_j, b_k$使得$b_ib_j \equiv_n b_ib_k$。这说明$n | b_i(b_j - b_k)$。
    由于$b_i$和$n$互素,根据倍和析因定理,存在整数$p, q$,使得:
    $$ b_ip + nq = 1.$$
    两边乘以$b_j - b_k$,就得到:
    $$ b_i(b_j - b_k)p + nq(b_j - b_k) = b_j - b_k.$$
    等式左边是$n$的倍数,因此$b_j$和$b_k$模$n$同余,这与它们的定义矛盾。\\
    因此命题的否定为假,原命题为真。
\end{proof2}

简化同余系的除法和整数不同,任何同余类都能整除另一个同余类,不需要余数、带余除法的概念。
每个同余类都有自己的“倒数”,比如在模$6$简化同余系中,$\overline{5}\times\overline{5} = \overline{1}$。
我们把同余类的“倒数”称为它的(乘法)\textbf{逆}。

\begin{xt}
    \mbox{}\\
    \indent 1. 写出模$12$的简化同余系。写出$\overline{7}^{12}$的逆。\\
    \indent 2. 比较模$12$简化同余系中的乘除法和模$4$完全同余系中的加减法,它们有何异同?\\
    \indent 3. 写出模$10$的简化同余系。写出$\overline{7}^{10}$的逆。\\
    \indent 4. 比较模$10$简化同余系中的乘除法和模$4$完全同余系中的加减法,它们有何异同?\\
    \indent 5. 给定素数$n$,写出模$n$简化同余系。
\end{xt}

\section{方余定理}
与模$n$简化同余系密切相关的一个定理是方余定理\footnote{这个定理也称为欧拉定理。但以欧拉命名的定理太多了。为了避免混淆,这里不采用。}。
\begin{tm}{\textbf{方余定理} }\label{tm:3-3-0}
    设$a$是模$n$简化同余系中某个同余类中的元素,则:
    $$ a^{\varphi(n)} \equiv_n 1 $$
    其中$\varphi(n)$是模$n$简化同余系中同余类的个数。
\end{tm}
比如,模$10$简化同余系有$4$个元素:$\overline{1}, \overline{3},\overline{7},\overline{9}$。
$7$属于同余类$\overline{7}$,则$7^4 \equiv_{10} 1$。

\begin{proof2}
    我们把模$n$简化同余系中的同余类用小于$n$且与$n$互素的正整数来代表,记为
    $$1 = b_1 < b_2 < \cdots < b_{\varphi(n)} = n-1.$$
    它们两两不同余。把它们各自乘以$a$,得到$\varphi(n)$个整数:$ab_1, ab_2, \cdots , ab_{\varphi(n)}$。
    前面我们已经证明了,它们仍然两两不同余。\\
    这说明这$\varphi(n)$个整数也分别代表模$n$简化同余系中的各个同余类。\\
    考虑乘积:$b_1 b_2 \cdots b_{\varphi(n)}$。$(ab_1) (ab_2) \cdots (ab_{\varphi(n)})$和它同余。
    也就是说:
    $$b_1 b_2 \cdots b_{\varphi(n)} \equiv_n (ab_1) (ab_2) \cdots (ab_{\varphi(n)}) \equiv_n a^{\varphi(n)} b_1 b_2 \cdots b_{\varphi(n)}.$$
    由于$b_1 b_2 \cdots b_{\varphi(n)}$也与$n$互素,我们把等式两边除以$b_1 b_2 \cdots b_{\varphi(n)}$,就得到:
    $$ a^{\varphi(n)} \equiv_n 1 . $$
\end{proof2}

如果$n$是素数,那么$1,2, \cdots , n-1$都和它互素,
于是模$n$的简化同余系就是$\{\overline{1},\overline{2}, \cdots , \overline{n-1}\}$,$\varphi(n) = n-1$。
根据方余定理,只要$a$不是$n$的倍数,就有:
$$ a^{n-1} \equiv_n 1 .$$
这个结论也叫做费马小定理。

\begin{xt}\label{xt:4-3-0}
    \mbox{}\\
    给定素数$n$,证明:\\
    \indent 1. 除了$\overline{1}$和$\overline{n-1}$,其它同余类的逆都不是自己。\\
    \indent 2. $(n-1)! \equiv_n -1.$ \\
    设$a$与$n$互素,称使得$a^m \equiv_n 1$的最小正整数$m$为$a$模$n$的\textbf{阶}。\\
    \indent 3. 证明$a$的阶整除$\varphi(n)$。\\
    \indent 4. 如果$a$的阶等于$\varphi(n)$,就说$a$是模$n$的\textbf{原根}。证明:如果$a$是模$n$的原根,
    那么模$n$简化同余系可以写成:$\{\overline{a^0}, \overline{a^1}, \cdots , \overline{a^{\varphi(n)-1}}\}$。\\
    \indent 5. 找出所有模$7$的原根。
\end{xt}


\chapter{用数据说话}
\section{样本和特征}
\section{描述和分析}
\section{数据的结构}

\chapter{数学和社会}
\section{随时代变化的数学}
\section{数学和科学}
\section{数学和现代化}

\end{document}