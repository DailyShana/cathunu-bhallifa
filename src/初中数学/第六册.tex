\documentclass[12pt,UTF8]{ctexbook}
\usepackage{ctex}
\usepackage{caption}
\usepackage{graphicx}
\usepackage{float}
\usepackage{wrapfig}
\usepackage{array}
\usepackage[table, dvipsnames, svgnames, x11names]{xcolor}
\usepackage{colortbl}% 
\usepackage{tabularx}
\usepackage{amsmath}
\usepackage{amssymb}
\usepackage{xfrac}
\usepackage{eucal}
\usepackage{titlesec}
\usepackage{amsthm}
\usepackage{tikz-cd}
\usepackage{enumitem}
\usepackage{verbatim}
\usepackage{fontspec,xunicode,xltxtra}
\usepackage{xeCJK} 

\definecolor{gl}{RGB}{246, 252, 240}
\definecolor{gd}{RGB}{236, 244, 230}
\definecolor{bg}{RGB}{242, 244, 228}


\setCJKmainfont[BoldFont=STZhongsong]{STSong}
\setCJKmonofont{simkai.ttf} % for \texttt
\setCJKsansfont{simfang.ttf} % for \textsf
\setlength\parskip{8pt}
\setlength{\fboxsep}{12pt}
\renewcommand\thesection{\arabic{chapter}.\arabic{section}}
\newtheorem{df}{定义}[section] 
\newtheorem{pp}{命题}[section]
\newtheorem{tm}{定理}[section]
\newtheorem{ex}{例子}[section]
\newtheorem{sk}{思考}[section]
\newtheorem{po}{公理}
\newtheorem*{so}{解答}
\newtheorem*{proof2}{证明}
\newtheorem{xt}{习题}[section]
\newtheorem{cor}{推论}[pp]
% 列举环境的行间距
\setenumerate[1]{itemsep=0pt,partopsep=0pt,parsep=0pt,topsep=0pt}
\setitemize[1]{itemsep=0pt,partopsep=0pt,parsep=0pt,topsep=0pt}
\setdescription{itemsep=0pt,partopsep=0pt,parsep=0pt,topsep=0pt}
\setlength{\intextsep}{2pt}%
\setlength{\columnsep}{2pt}%
% 新函数
\renewcommand\parallel{\mathrel{/\mskip-4mu/}}
% 章节字体大小
\titleformat{\section}{\zihao{-2}\bfseries}{ \thesection }{16pt}{}
% 封面
\title{\zihao{0} \bfseries 第六册}
\author{\zihao{2} \texttt{大青花鱼}}
% \date{\bfseries\today}
\date{}
% 正文
\begin{document}
\maketitle
\tableofcontents
\newpage

\chapter{代数式的关系}
\section{代数恒等式}
\section{代数不等式}

\chapter{三段论(下)}
\section{三段论的规则}
\section{三段论的应用}

\chapter{投影和视图}
\section{平面和立体}
\section{三视图}
\section{表面的展开}

\chapter{同余}
\begin{ex}\label{ex:3-0-0}
    $7^{65}$的个位数是多少?
\end{ex}
\begin{so}
    从$7^0,7^1,7^2,7^3\cdots$开始找规律。$7^0=1$,$7^1=7$,$7^2=49$,$7^3=343$,$7^4=2401$,$7^5=16807$。
    $7^4$和$7^0$的个位数都是$1$,$7^5$和$7^1$的个位数都是$7$。我们可以总结出这样的规律:个位数是$1$的,乘以$7$得到$7$;
    个位数是$7$的,乘以$7$得到$9$;个位数是$9$的,乘以$7$得到$3$;个位数是$3$的,乘以$7$得到$1$。

    也就是说,如果把$7^0,7^1,7^2,7^3\cdots$的个位数写成一列,应该是这个样子的:
    $$ 1, 7, 9, 3, 1, 7, 9, 3, 1, 7, \cdots$$
    用归纳法不难证明,这列数字以$4$为周期不断重复。所以,要求$7^{65}$的个位数,可以看$65$在相关的周期里处于哪个位置。
    换句话说,只要看$65$除以$4$的余数。$65 = 16 \times 4 + 1$,所以$7^{65}$的个位数和$7^1$的个位数一样,都是$7$。
\end{so}

从这个例子可以看出,两个整数除以同一个数得到相同的余数,是一个重要的性质。我们把这种性质称为\textbf{同余}。
比如,$65$和$1$除以$4$余数都是$1$,我们就说$65$和$1$模$4$同余。$7^{65}$和$7^1$除以$10$余数都是$7$,
我们说$7^{65}$和$7^1$模$10$同余,记为:
$$ 7^{65} \equiv_{10} 7^1 $$

\section{同余类}
整数除以$3$,余数有$0,1,2$三种可能。整数除以$10$,余数有$0,1,\cdots , 9$十种可能。
一般来说,给定正整数$n$,整数除以$n$,余数有$0,1,\cdots , n-1$这$n$种可能。
因此,按除以$n$的余数,可以把整数集分成$n$类。同属一类的数,模$n$同余,所以这$n$类数叫作模$n$\textbf{同余类}。
所有模$n$同余类的集合,叫作模$n$同余系。

每个模$n$同余类,可以写成$\{kn + a \, | \, k\in\mathbb{Z} \}$的形式。也就是说,可以看成某个数$a$不断加上或减去$n$得到的所有数的集合。这个集合是无穷的。不同的模$n$同余类,交集是空集,并集是$\mathbb{Z}$。也就是说,它们是$\mathbb{Z}$的分划。

为了方便,我们从每个模$n$同余类中选一个元素,代表这个同一类。一般来说,可以选$0,1,\cdots,n-1$个数。我们给它们加个上划线,以和作为整数的$0,1,\cdots,n-1$区分:
$$\overline{0},\overline{1},\cdots,\overline{n-1}$$

如果要强调$n$,可以把$n$加在右上角:
$$\overline{0}^n,\overline{1}^n,\cdots,\overline{n-1}^n$$

给定整数$m$,我们可以把它对应到某个模$n$同余类,称为对$n$\textbf{取模}。
比如$n=5$时,$24 \equiv_5 4$,我们把$24$对应到$\overline{4}^5$,
或者说,$24$对$5$取模,得$\overline{4}^5$。

同余关系和相等关系很像,它们是否有一样的性质呢?我们可以验证,同余关系满足以下的性质:
\begin{enumerate}
    \item $\forall \,\, a\in \mathbb{Z}, \quad a \equiv_n a$;
    \item $\forall \,\, a, b \in \mathbb{Z}$,如果$a \equiv_n b$,那么$b \equiv_n a$;
    \item $\forall \,\, a, b \in \mathbb{Z}$,如果$a \equiv_n b$,$b \equiv_n c$,那么$a \equiv_n c$。
\end{enumerate}

满足以上三个性质的二元关系(两个元素之间的关系)称为等价关系。数与数的等于关系是等价关系,数与数的同余关系
也是等价关系。因此,我们可以把同余关系用作同余类之间的等于关系。

整数之间有四则运算,模$n$同余类之间,也可以进行运算。以$n=5$为例子。
我们分别计算$24$和$37$除以$5$的余数,以及它们的和$61$除以$5$的余数:
$$ 24 \equiv_5 4, \,\,\, 37 \equiv_5 2 , \,\,\, 61 \equiv_5 1$$

可以发现:$ 4 + 2 \equiv_5 1$,也就是说,取模和加法可以交换顺序。
可以验证,两个同余类中各取一个元素相加,和所在的同余类,就是两者取模后的和所在的同余类。
用集合的语言,可以写成:
$$\{kn + a + ln + b \, | \, k\in\mathbb{Z}, \, l\in\mathbb{Z} \} = \{kn + a + b \, | \, k\in\mathbb{Z} \}$$

所以,可以定义同余类的加法:
$$ \overline{a} + \overline{b} = \overline{a + b}$$

其中的$\overline{a + b}$指的是$a+b$所在的同余类。为了方便,我们用$a + b$作为代表。

可以验证,同余类的加法也满足结合律和交换律。这里我们只证明同余类的加法满足结合律:

\begin{proof2}
    由上可知$ \overline{a} + \overline{b} = \overline{a + b}$,所以
    $$ (\overline{a} + \overline{b}) + \overline{c} = \overline{a + b}+ \overline{c} = \overline{a + b + c}.$$
    类似可得:
    $$ \overline{a} + (\overline{b} + \overline{c}) = \overline{a}+ \overline{b + c} = \overline{a + b + c}.$$
    于是
    $$ (\overline{a} + \overline{b}) + \overline{c}  = \overline{a + b + c} = \overline{a} + (\overline{b} + \overline{c}).$$
\end{proof2}

类似可以定义同余类的减法和乘法:
$$ \overline{a} - \overline{b} = \overline{a - b}, \,\,\, \overline{a} \cdot \overline{b} = \overline{a \cdot b}$$

可以验证,同余类的减法性质和整数减法一样,同余类的乘法也满足结合律、交换律和分配律。

能否定义同余类的除法呢?我们来看一个例子。设$n=6$,考虑等式$12 \div 4 = 3$。
$12$、$4$和$3$对$6$取模,得到$0$、$4$和$3$。考虑等式$60 \div 10 = 6$。$60$、$10$和$6$对$6$取模,
得到$0$、$4$和$0$。也就是说,两个模$6$同余类中各取元素相除,商所在的同余类不是唯一的。
所以,我们没法定义模$6$同余类的除法。

再看另一个例子。设$n=5$,考虑以下的“乘法表”:
\begin{center}
    \begin{tabular}{ | p{2em}<{\centering} | p{2em}<{\centering} | p{2em}<{\centering} | p{2em}<{\centering} | p{2em}<{\centering} | p{2em}<{\centering} | }
        \hline
            $\times$   & $\overline{0}$ & $\overline{1}$ & $\overline{2}$ & $\overline{3}$ & $\overline{4}$ \\ [0.5ex] 
        \hline
        $\overline{0}$ & $\overline{0}$ & $\overline{0}$ & $\overline{0}$ & $\overline{0}$ & $\overline{0}$ \\  
        \hline
        $\overline{1}$ & $\overline{0}$ & $\overline{1}$ & $\overline{2}$ & $\overline{3}$ & $\overline{4}$ \\
        \hline
        $\overline{2}$ & $\overline{0}$ & $\overline{2}$ & $\overline{4}$ & $\overline{1}$ & $\overline{3}$ \\
        \hline
        $\overline{3}$ & $\overline{0}$ & $\overline{3}$ & $\overline{1}$ & $\overline{4}$ & $\overline{2}$ \\
        \hline 
        $\overline{4}$ & $\overline{0}$ & $\overline{4}$ & $\overline{3}$ & $\overline{2}$ & $\overline{1}$ \\
        \hline
    \end{tabular}
\end{center}

可以看出,任何模$5$同余类乘以$\overline{0}$都得到$\overline{0}$,非$\overline{0}$同余类乘以不同的同余类,结果也不同。
这说明每个同余类除以另一个同余类(非$\overline{0}$),都必然有唯一的结果。这样我们就定义了模$5$同余系里的除法。

\begin{xt}\label{xt:3-0-0}
    \mbox{}\\
    动手做一做:\\
    \indent 1. 证明同余关系满足等价关系所要求的三个性质。 \\
    \indent 2. 证明同余类的加法满足交换律。 \\
    \indent 3. 证明同余类的减法是加法的逆运算。\\
    \indent 4. 证明同余类的乘法满足结合律和交换律。\\
    \indent 5. 证明同余类的乘法满足分配律。\\
    \indent 6. 分别画出模$3$同余系和模$4$同余系的“乘法表”。它们和模$5$同余系的“乘法表”哪些地方相同,哪些地方不同?
\end{xt}

\section{完全同余系和简化同余系}
上一节我们提到模$6$同余系无法定义除法,而模$5$同余系可以定义除法。两者有什么不同呢?
我们画出模$6$同余系的“乘法表”:
\begin{center}
    \begin{tabular}{ | p{2em}<{\centering} | p{2em}<{\centering} | p{2em}<{\centering} | p{2em}<{\centering} | p{2em}<{\centering} | p{2em}<{\centering} | p{2em}<{\centering} | }
        \hline
            $\times$   & $\overline{0}$ & $\overline{1}$ & $\overline{2}$ & $\overline{3}$ & $\overline{4}$ & $\overline{5}$ \\ [0.5ex] 
        \hline
        $\overline{0}$ & $\overline{0}$ & $\overline{0}$ & $\overline{0}$ & $\overline{0}$ & $\overline{0}$ & $\overline{0}$ \\  
        \hline
        $\overline{1}$ & $\overline{0}$ & $\overline{1}$ & $\overline{2}$ & $\overline{3}$ & $\overline{4}$ & $\overline{5}$ \\
        \hline
        $\overline{2}$ & $\overline{0}$ & $\overline{2}$ & $\overline{4}$ & $\overline{0}$ & $\overline{2}$ & $\overline{4}$ \\
        \hline
        $\overline{3}$ & $\overline{0}$ & $\overline{3}$ & $\overline{0}$ & $\overline{3}$ & $\overline{0}$ & $\overline{3}$ \\
        \hline 
        $\overline{4}$ & $\overline{0}$ & $\overline{4}$ & $\overline{2}$ & $\overline{0}$ & $\overline{4}$ & $\overline{2}$ \\
        \hline
        $\overline{5}$ & $\overline{0}$ & $\overline{5}$ & $\overline{4}$ & $\overline{3}$ & $\overline{2}$ & $\overline{1}$ \\
        \hline
    \end{tabular}
\end{center}
可以看到,这个“乘法表”和模$5$同余系的大有不同。同一行或同一列常有重复。
这说明不同的同余类乘同一个同余类得到同一个结果。比如
$$\overline{2}\times \overline{4} = \overline{5}\times \overline{4} = \overline{2}. $$
这就使我们没法定义除法。

如果我们把上面的等式稍作变化,会得到:
$$\overline{0} = (\overline{5} - \overline{2})\times \overline{4} = \overline{3} \times \overline{4}.$$
也就是说,有非$\overline{0}$的同余类相乘等于$\overline{0}$。
同余类乘法的这个性质和整数乘法完全不同。我们把这种非$\overline{0}$同余类叫做\textbf{零因子}。
整数中没有零因子:非$0$的整数相乘必然不是$0$。而只要有这种零因子存在,同余系中就会发生“不同的同余类乘同一个同余类得到同一个结果”的现象,
从而无法定义除法。

% 另一个使模$6$同余系没法定义除法的问题是“倒数”的问题。举例来说,没有同余类乘以$\overline{4}$等于$\overline{1}$。
% 因此,我们无法确定$\overline{1}\div\overline{4}$。也许有人会说,我们可以定义“带余除法”。
% 但定义带余除法首先需要定义大于和小于关系,而同余系中并没有大于和小于关系。

有什么办法在模$6$同余系中定义除法呢?我们可以选一部分同余类,在其中定义除法。
如果同余类$\overline{a}$的代表$a$与$6$不互素,设最大公因数是$b$,那么
$$ \frac{a}{b} \times 6 = a \times \frac{6}{b} $$
于是有$\overline{a} \times \overline{\frac{6}{b}} = \overline{0}$,出现零因子。
因此,为了避免零因子问题,我们只选和$6$互素的数所在的同余类,也就是$\overline{1}$和$\overline{5}$。
我们发现$\{\overline{1}, \overline{5}\}$中可以定义乘法和除法(但不再满足加减法)。
我们把模$6$同余系称为模$6$的\textbf{完全同余系},
把$\{\overline{1}, \overline{5}\}$称为模$6$的\textbf{简化同余系}。

一般来说,我们把模$n$同余系称为模$n$的完全同余系,在其中可以定义加减法和乘法;
把其中所有和$n$互素的同余类的集合称为模$n$的简化同余系,在其中可以定义乘法和除法
\footnote{通常不把$\overline{0}$计入简化剩余系,以省去讨论除以$\overline{0}$的问题。}。

简化同余系的除法和整数不同之处是,任何同余类都能整除另一个同余类,不需要余数、带余除法的概念。
每个同余类都有自己的“倒数”,比如在模$6$简化同余系中,$\overline{5}\times\overline{5} = \overline{1}$。
我们把同余类的“倒数”称为它的(乘法)\textbf{逆}。

\begin{xt}
    \mbox{}\\
    \indent 1. 写出模$12$的简化同余系。写出$\overline{7}^{12}$的逆。\\
    \indent 2. 比较模$12$简化同余系中的乘除法和模$4$完全同余系中的加减法,它们有何异同?\\
    \indent 3. 写出模$10$的简化同余系。写出$\overline{7}^{10}$的逆。\\
    \indent 4. 比较模$10$简化同余系中的乘除法和模$4$完全同余系中的加减法,它们有何异同?\\
    \indent 5. 证明简化同余系中的乘法满足消去律:如果$\overline{ab} \equiv \overline{ac}$,
    那么$\overline{b} \equiv \overline{c}$。
\end{xt}

\section{方余定理}
与模$n$简化同余系密切相关的一个定理是方余定理。
\begin{tm}{\textbf{方余定理} }\label{tm:3-3-0}
    设$a$是模$n$简化同余系中某个同余类中的元素,则:
    $$ a^{\varphi(n)} \equiv_n 1 $$
    其中$\varphi(n)$是模$n$简化同余系中同余类的个数。
\end{tm}
比如,模$10$简化同余系有$4$个元素:$\overline{1}, \overline{3},\overline{7},\overline{9}$。
$7$属于同余类$\overline{7}$,则$7^4 \equiv_{10} 1$。

\begin{proof2}
    从模$n$简化同余系中的每个同余类中选一个代表$b_1, b_2, \cdots , b_{\varphi(n)}$,它们两两不同余。
    把它们各自乘以$a$,得到$\varphi(n)$个整数:$ab_1, ab_2, \cdots , ab_{\varphi(n)}$。它们仍然两两不同余。\\
    用反证法来证明。反设其中某两个数模$n$同余,不妨设$ab_1$和$ab_2$模$n$同余:
    $$ ab_1 \equiv_n ab_2.$$
    这说明$n | a(b_1 - b_2)$。由于$a$和$n$互素,根据倍和析因定理,存在整数$p, q$,使得:
    $$ ap + nq = 1.$$
    两边乘以$b_1 - b_2$,就得到:
    $$ a(b_1 - b_2)p + nq(b_1 - b_2) = b_1 - b_2.$$
    等式左边是$n$的倍数,因此$b_1$和$b_2$模$n$同余,这与$b_1, b_2, \cdots , b_{\varphi(n)}$的选取方式矛盾。\\
    因此命题的否定为假,原命题为真:$ab_1, ab_2, \cdots , ab_{\varphi(n)}$两两不同余。
    这说明这$\varphi(n)$个整数也可以分别代表模$n$简化同余系中的各个同余类。\\
    考虑这样一个数:“小于$n$的正整数中所有与$n$互素的数的乘积”。$b_1 b_2 \cdots b_{\varphi(n)}$和
    $(ab_1) (ab_2) \cdots (ab_{\varphi(n)})$都和这个数同余。也就是说:
    $$b_1 b_2 \cdots b_{\varphi(n)} \equiv_n (ab_1) (ab_2) \cdots (ab_{\varphi(n)}) \equiv_n a^{\varphi(n)} b_1 b_2 \cdots b_{\varphi(n)}.$$
    由于$b_1 b_2 \cdots b_{\varphi(n)}$也与$n$互素,我们可以把等式两边除以$b_1 b_2 \cdots b_{\varphi(n)}$,就得到:
    $$ a^{\varphi(n)} \equiv_n 1 .$$
\end{proof2}

\begin{xt}\label{xt:4-3-0}
    设$a$与$n$互素,称使得$a^m \equiv_n 1$的最小正整数$m$为$a$模$n$的\textbf{阶}。\\
    \indent 1. 证明$a$的阶整除$\varphi(n)$。\\
    \indent 2. 如果$a$的阶等于$\varphi(n)$,就说$a$是模$n$的\textbf{原根}。证明:如果$a$是模$n$的原根,
    那么模$n$简化同余系可以写成:$\{\overline{a^0}, \overline{a^1}, \cdots , \overline{a^{\varphi(n)-1}}\}$。\\
    \indent 3. 找出所有模$7$的原根。
\end{xt}


\chapter{用数据说话}
\section{样本和特征}
\section{描述和分析}
\section{数据的结构}

\chapter{数学和社会}
\section{随时代变化的数学}
\section{数学和科学}
\section{数学和现代化}

\end{document}