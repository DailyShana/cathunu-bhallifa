\documentclass[12pt,UTF8]{ctexbook}

% 导入设定
% File settings - applied to all
% 导入第三方库
\usepackage{ctex}
\usepackage{array}
\usepackage{graphicx}
\usepackage{wrapfig}
\usepackage[table,dvipsnames]{xcolor}
\usepackage{tabularx}
\usepackage{longtable}
\usepackage{float}
\usepackage{amsmath}
\usepackage{amssymb}
\usepackage{mathtools}
\usepackage{polynom}
\usepackage{xfrac}
\usepackage{eucal}
\usepackage{titlesec}
\usepackage{amsthm}
\usepackage{mhchem}
\usepackage{tikz-cd}
\usepackage{enumitem}
\usepackage{verbatim}
\usepackage[makeroom]{cancel}
\usepackage[toc,page]{appendix}
\usepackage{fontspec,xunicode,xltxtra}
\usepackage{xeCJK} 
\usepackage{caption}
\usepackage[b]{esvect}
\usepackage{thmtools, thm-restate}
\usepackage{pifont}
\usepackage[perpage,symbol*]{footmisc}

% 修改脚注的编号为加圈样式,并且各页单独编号
\DefineFNsymbols{circled}{{\ding{192}}{\ding{193}}{\ding{194}}
{\ding{195}}{\ding{196}}{\ding{197}}{\ding{198}}{\ding{199}}{\ding{200}}{\ding{201}}}
\setfnsymbol{circled}

% 自定义颜色
\definecolor{gl}{RGB}{246, 252, 240}
\definecolor{gd}{RGB}{236, 244, 230}
\definecolor{bg}{RGB}{242, 244, 228}

% 定义字体
\setCJKmainfont[BoldFont=STZhongsong]{STSong}  % 普通字体、粗体
\setCJKmonofont{simkai.ttf} % \texttt
\setCJKsansfont{simfang.ttf} % \textsf

% 自制命令
\renewcommand{\thesection}{\arabic{chapter}.\arabic{section}}  % 章节使用阿拉伯数字
\renewcommand{\parallel}{\mathrel{/\mskip-4mu/}}  % 平行符号
\renewcommand{\proofname}{\indent\bf 证明}  % 自定义证明标题
\renewcommand{\qedsymbol}{\hfill$\square$}  % 自定义证毕符号
\newcommand{\e}{\mathrm{e}}  % 自然底数
\newcommand{\dash}{\,–\,}  % 短折号
\newcommand{\tong}[1]{\overset{#1}{\equiv\joinrel\equiv}}  % 同余等号
\newcommand{\di}[1]{\,\mathrm{d}#1}  % 微元d
\newcommand{\qu}[2]{\displaystyle\left(#1;#2\right)}  % 开区间
% 局部展开 developpements limites
\newcommand{\oveq}[1]{\overset{#1}{=}}   % equal over
\newcommand{\olim}[1]{\mathit{o}\left(#1\right)}  % petit o
\newcommand{\Olim}[1]{\mathcal{O}\left(#1\right)}  % grand O
\newcommand{\Tlim}[1]{\mathcal{\Theta}\left(#1\right)}  % grand theta
\newcommand{\eqlim}[1]{\overset{#1}{\sim}}  % equivalence
\newcommand{\vect}[1]{\left\langle #1 \right\rangle}  % 生成空间 generated space

\newcommand{\arccot}{\operatorname{arccot}}  % 反余弦函数
\newcommand{\dlim}[1]{^{\color{gray}\prime}#1}  % 数字分隔符
\newcommand{\lian}[1]{  % 极限符号
    \underset{#1}{\operatorname{lian}\,}
}
\newcommand{\nji}[2]{\displaystyle\left( #1 \,|\, #2 \right)}  % 内积
\newcommand{\dangle}{  % 角符号
    \mathord{
        \text{  %
            \tikz[baseline] \draw (0.8em,0ex) -- (0.3em, 0ex) -- (.6em, 1.5ex) -- (.8em, 1.5ex) -- (.5em, 0ex) -- cycle;
        }
    }
}
\newcommand{\xangle}{  % 角符号
    \mathord{
        \text{%
        \tikz[baseline] \draw (0.8em,1.5ex) -- (0.3em, 0ex) -- (.64em, 0ex) -- (.8em, .36ex) -- (.42em, .36ex) -- cycle;
        }
    }
}
\newcommand{\bu}{  % 补集符号
    \mathbin{
        \text{
            \tikz[baseline=-0.6ex]{
                \node[draw, fill=black, minimum size=0.8ex, inner sep=0pt, rectangle] (bu) {};
                \node[draw=none, fill=white, minimum size=0.6ex, inner sep=0pt, circle] at (bu.center) {};
            }
        }
    }
}
\newcommand{\rectbx}{  % 长方形符号
    \mathord{
        \text{%
            \tikz[baseline] \draw (0,.1ex) -- (.4em,.1ex) -- (.4em,1.5ex) -- (0em,1.5ex) -- cycle;
        }
    }
}
\newcommand{\tr}{  % 矩阵转置符号 A^{\tr} 
    \mathord{
        \begin{tikzpicture}[baseline=-0.2em, line width=0.3pt]
        \draw (-0.15em, 0.15em) -- (0.06em, -0.06em);
        \draw (45:0.15em) arc[start angle=45, end angle=225, radius=0.15em];
    \end{tikzpicture}
    }
}
\newcommand{\arcangle}{\mathord{\mathpalette\doarcangle\relax}}  % 带弧的角度符号 - 交角
\newcommand{\doarcangle}[2]{  % 
    \hbox{%
        \sbox0{$#1B$}%
        \sbox2{$#1<$}%
        \raisebox{\dimexpr\dp0+(\ht0-\ht2)/2}{%
            $#1<\mspace{-9mu}\mathrel{)}\mspace{2mu}$%
        }%
    }%
}
\newcommand{\parasbx}{  % 平行四边形符号
    \mathord{
        \text{%
            \tikz[baseline] \draw (0,.1ex) -- (.8em,.1ex) -- (1em,1.6ex) -- (.2em,1.6ex) -- cycle;
        }
    }
}
\usetikzlibrary{calc,topaths}
\newcommand{\widearc}[1]{  % 可伸缩圆弧符号
    \tikz[baseline=(wideArcAnchor.base)]{
        \node[inner sep=0] (wideArcAnchor) {$#1$}; 
        \coordinate (wideArcAnchorA) at ($(wideArcAnchor.north west) + (0.15em,0.1em)$);
        \coordinate (wideArcAnchorB) at ($(wideArcAnchor.north east) + (0.0em,0.1em)$);
        \draw[line width=0.1ex,line cap=round,out=45,in=135] (wideArcAnchorA) to (wideArcAnchorB);
    }
}

% 定义、定理、证明等块环境
\theoremstyle{definition}
\newtheorem{df}{定义}[section] 
\newtheorem*{po}{公理}
\newtheorem{pp}{命题}[section]
\newtheorem{tm}{定理}[section]
\newtheorem{cor}{推论}[pp]
\newtheorem{ex}{例子}[section]
\newtheorem{et}{例题}[section]
\newtheorem*{ex*}{例子}
\newtheorem*{so}{解答}
\theoremstyle{plain}
\newtheorem{sk}{思考}[section]
\newtheorem{xt}{习题}[section]
\renewenvironment{proof}{\paragraph{\textbf{证明:}}}{\hfill$\square$}
% \declaretheorem[name=定义, numberwithin=section, shaded={rulecolor={rgb}{0.1,0.7,0.4},
% rulewidth=2pt, bgcolor={rgb}{0.96,1,0.99}}]{df}
% \declaretheorem[name=定理, numberwithin=section, shaded={rulecolor={rgb}{0.1,0.4,0.7},
% rulewidth=2pt, bgcolor={rgb}{0.96,0.99,1}}]{tm}
% \declaretheorem[name=思考, numberwithin=section, shaded={rulecolor={rgb}{0,0.7,0.7},
% rulewidth=2pt, bgcolor={rgb}{0.98,1,1}}]{sk}
% \declaretheorem[name=习题, numberwithin=section, shaded={rulecolor={rgb}{0.91,0.84,0.42},
% rulewidth=2pt, bgcolor={rgb}{1,0.98,0.93}}]{xt}

\setlength{\intextsep}{2pt}%
\setlength{\columnsep}{2pt}%
% 列举环境
\setlist{label=\textbullet}
% 列举环境行间距
\setenumerate[1]{itemsep=0pt,partopsep=0pt,parsep=0pt,topsep=0pt}
\setitemize[1]{itemsep=0pt,partopsep=0pt,parsep=0pt,topsep=0pt}
\setdescription{itemsep=0pt,partopsep=0pt,parsep=0pt,topsep=0pt}
% 章节间距
\setlength\parskip{8pt}
% 文本框间距
\setlength{\fboxsep}{12pt}
% 章节字体大小
\titleformat{\section}{\zihao{-2}\bfseries}{ \thesection }{16pt}{}

% 封面
\title{\zihao{0} \bfseries 小学数学基础知识}
\author{\zihao{2} \texttt{大青花鱼}}
% \date{\bfseries\today}
\date{}
% 正文
\begin{document}

\maketitle
\tableofcontents
\newpage

\chapter{数学基础}

为初中数学学习准备的算学和形学基础知识。和[思理入门](../../悟数学/思理入门.pdf)一起作为学习的基础。



\begin{center}
    \fbox{
        \shortstack[l]{
            \textbf{注意:本册不是教材,仅是知识清单。}
        }
    }
\end{center}

\section{自然数和数字}

\textbf{自然数}是自然诞生的数,源自数数(或者叫计数)。从\( 0 \)开始,不断加\( 1 \),就能得到任何自然数。

数的表示:我们用\( 0 \)、\( 1 \)、\( 2 \)、\( 3 \)等数字记录自然数。\textbf{阿拉伯数字}:
\[ 0,1,2,3,4,5,6,7,8,9 \]

数的大小:能通过加\( 1 \)从一个数得到另一个数,就说它比另一个数小,另一个数比它大。

我们采用\textbf{十进位制}表示数。十进制位的名字:个、十、百、千、万、十万、百万、千万、亿、十亿、……

\subsection{十进制自然数比较大小}

\begin{itemize}
\item 位数多的大,位数少的小。
\item 位数相同的,对齐后从高到低比较。遇到第一个不同的,数字大的大,数字小的小。
\item 没有不同,则两数相等。
\end{itemize}

\section{四则运算}

四则运算指\textbf{加减乘除}。

\subsection{称呼}

\begin{itemize}
\item 加法:被加数 + 加数 = 和
\item 乘法:被乘数 × 乘数 = 积
\item 减法:被减数 - 减数 = 差
\item 除法:被除数 ÷ 除数 = 商
\end{itemize}

\(  2 + 3 = 5  \),\( 2 \)加\( 3 \)等于\( 5 \)。

\(  2 \times 3 = 6  \),\( 3 \)乘\( 2 \)等于\( 5 \),\( 2 \)乘以\( 3 \)等于\( 5 \)。

\(  5 - 3 = 2  \),\( 5 \)减\( 3 \)等于\( 2 \)。

\(  6 \div 3 = 2  \),\( 3 \)除\( 6 \)等于\( 2 \),\( 6 \)除以\( 3 \)等于\( 2 \)。

\subsection{基本法则}

\begin{itemize}
\item 加法交换律:交换加数与被加数,和不变。
\item 乘法交换律:交换乘数与被乘数,积不变。
\item 任何数加\( 0 \)等于自己。
\item 任何数乘\( 1 \)等于自己。
\item 任何数乘\( 0 \)等于\( 0 \)。
\item 加法结合律:任意三个数相加,先把前两个数相加,或者先把后两个数相加,和不变。
\item 乘法结合律:任意三个数相乘,先把前两个数相乘,或者先把后两个数相乘,积不变。
\item 乘法对加法的分配律:两个数分别乘同一个数再相加,等于先相加再乘这个数。
\item 减法是加法的逆运算。
\item 除法是乘法的逆运算。
\item 不能除以\( 0 \)。
\end{itemize}

\subsection{运算顺序}

\begin{itemize}
\item 乘除法比加减法优先。
\item 括号可以改变运算顺序。
\item 先加后减,等于先减后加。
\item 先乘后除,等于先除后乘。
\item 减差等于先减后加。
\item 除商等于先除后乘。
\end{itemize}

\section{自然数的运算}

\subsection{自然数加法的计算}

使用竖式计算自然数的加法。加数可以是两个数,也可以多于两个数。

\begin{itemize}
\item 被加数在上,加数在下,个位对齐。
\item 从个位开始,从低到高计算。
\item 把加数的个位数加起来。每超过十,就减去十,同时在下一位的进位栏添上一。把结果写在和的个位上。
\item 把加数的十位数以及进位栏里的所有一加起来。每超过十,就减去十,同时在下一位的进位栏添上一。把结果写在和的十位上。
\item 依此类推,把加数的某位数以及它的进位栏里的所有一加起来。每超过十,就减去十,同时在下一位的进位栏添上一。把结果写在和的这位上。
\item 只要某位上有加数,或它的进位栏里有一,就需要把它们加起来。缺少加数的话,用〇补上。直到某位上既没有加数,进位栏里也没有一,就结束。
\end{itemize}

\subsection{自然数减法的计算}

使用竖式计算两个自然数的减法。

\begin{itemize}
\item 被减数在上,减数在下,个位对齐。
\item 从个位开始,从低到高计算。
\item 用被减数的个位减去减数的个位。
\item 如果被减的数不小于减数,那么把差写到差的个位上,转到下一位。如果被减的数比减数小,那么从下一位借一当十,把十加上被减的数后减去减数的差写在差的个位上。在下一位的借位栏里加一。
\item 用被减数的十位减去减数的十位。如果借位栏里有一,则先将减数加一。
\item 如果被减的数不小于减数,那么把差写到差的十位上,转到下一位。如果被减的数比减数小,那么从下一位借一当十,把十加上被减的数后减去减数的差写在差的十位上。在下一位的借位栏里加一。
\item 依此类推,用被减数的某位减去减数的同位。如果它的借位栏里有一,则先将减数加一。
\item 如果被减的数不小于减数,那么把差写到差的这位上,转到下一位。如果被减的数比减数小,那么从下一位借一当十,把十加上被减的数后减去减数的差写在差的这位上。在下一位的借位栏里加一。
\item 只要某位上有被减数或减数,或它的进位栏里有一,就需要执行操作。缺少减数或被减数的话,用〇补上。直到某位上既没有被减数,也没有减数,进位栏里也没有一,就结束。
\end{itemize}

\subsection{自然数乘法的计算}

\subsubsection{九九乘法表}

\begin{center}
    
    {\LARGE \textbf{九九乘法表}}

    \begin{tabular}{ | p{2em}<{\centering} | p{2em}<{\centering} | p{2em}<{\centering} |  p{2em}<{\centering} | p{2em}<{\centering} | p{2em}<{\centering} | p{2em}<{\centering} | p{2em}<{\centering} | p{2em}<{\centering} | p{2em}<{\centering} |}
        \hline
        \cellcolor[RGB]{250,250,250}$\times$ & \phantom{0}1 & \phantom{0}2 & \phantom{0}3 & \phantom{0}4 & \phantom{0}5 & \phantom{0}6 & \phantom{0}7 & \phantom{0}8 & \phantom{0}9  \\  
        \hline
        1 & \cellcolor[RGB]{252,239,236}\phantom{0}1 & \cellcolor[RGB]{252,243,236}\phantom{0}2 & \cellcolor[RGB]{250,248,239}\phantom{0}3 & \cellcolor[RGB]{248,251,242}\phantom{0}4 & \cellcolor[RGB]{245,251,245}\phantom{0}5 & \cellcolor[RGB]{242,248,248}\phantom{0}6 & \cellcolor[RGB]{239,245,250}\phantom{0}7 & \cellcolor[RGB]{236,242,252}\phantom{0}8 & \cellcolor[RGB]{236,239,252}\phantom{0}9\\
        \hline
        2 & \cellcolor[RGB]{252,226,220}\phantom{0}2 & \cellcolor[RGB]{252,234,220}\phantom{0}4 & \cellcolor[RGB]{248,243,226}\phantom{0}6 & \cellcolor[RGB]{243,249,232}\phantom{0}8 & \cellcolor[RGB]{237,249,237}10 & \cellcolor[RGB]{232,244,243}12 & \cellcolor[RGB]{226,237,248}14 & \cellcolor[RGB]{220,231,252}16 & \cellcolor[RGB]{220,226,252}18\\
        \hline
        3 & \cellcolor[RGB]{253,214,205}\phantom{0}3 & \cellcolor[RGB]{253,226,205}\phantom{0}6 & \cellcolor[RGB]{247,239,214}\phantom{0}9 & \cellcolor[RGB]{239,248,222}12 & \cellcolor[RGB]{231,248,231}15 & \cellcolor[RGB]{222,241,239}18 & \cellcolor[RGB]{214,231,247}21 & \cellcolor[RGB]{205,221,253}24 & \cellcolor[RGB]{205,214,253}27\\
        \hline
        4 & \cellcolor[RGB]{253,200,189}\phantom{0}4 & \cellcolor[RGB]{253,216,189}\phantom{0}8 & \cellcolor[RGB]{245,234,201}12 & \cellcolor[RGB]{234,246,212}16 & \cellcolor[RGB]{223,246,223}20 & \cellcolor[RGB]{212,236,234}24 & \cellcolor[RGB]{201,223,245}28 & \cellcolor[RGB]{189,210,253}32 & \cellcolor[RGB]{189,201,253}36\\
        \hline
        5 & \cellcolor[RGB]{253,187,173}\phantom{0}5 & \cellcolor[RGB]{253,207,173}10 & \cellcolor[RGB]{243,230,187}15 & \cellcolor[RGB]{229,244,201}20 & \cellcolor[RGB]{215,244,215}25 & \cellcolor[RGB]{201,232,229}30 & \cellcolor[RGB]{187,215,243}35 & \cellcolor[RGB]{173,200,253}40 & \cellcolor[RGB]{173,188,253}45\\
        \hline
        6 & \cellcolor[RGB]{254,175,158}\phantom{0}6 & \cellcolor[RGB]{254,199,158}12 & \cellcolor[RGB]{242,226,175}18 & \cellcolor[RGB]{226,243,192}24 & \cellcolor[RGB]{209,243,209}30 & \cellcolor[RGB]{192,229,226}36 & \cellcolor[RGB]{175,209,242}42 & \cellcolor[RGB]{158,190,254}48 & \cellcolor[RGB]{158,176,254}54\\
        \hline
        7 & \cellcolor[RGB]{254,161,142}\phantom{0}7 & \cellcolor[RGB]{254,190,142}14 & \cellcolor[RGB]{240,221,162}21 & \cellcolor[RGB]{221,241,182}28 & \cellcolor[RGB]{201,241,201}35 & \cellcolor[RGB]{182,224,221}42 & \cellcolor[RGB]{162,201,240}49 & \cellcolor[RGB]{142,179,254}56 & \cellcolor[RGB]{142,163,254}63\\
        \hline
        8 & \cellcolor[RGB]{255,149,127}\phantom{0}8 & \cellcolor[RGB]{255,181,127}16 & \cellcolor[RGB]{239,217,150}24 & \cellcolor[RGB]{217,240,172}32 & \cellcolor[RGB]{195,240,195}40 & \cellcolor[RGB]{172,221,217}48 & \cellcolor[RGB]{150,194,239}56 & \cellcolor[RGB]{127,169,255}64 & \cellcolor[RGB]{127,151,255}72\\
        \hline
        9 & \cellcolor[RGB]{255,136,111}\phantom{0}9 & \cellcolor[RGB]{255,172,111}18 & \cellcolor[RGB]{237,213,137}27 & \cellcolor[RGB]{212,238,162}36 & \cellcolor[RGB]{187,238,187}45 & \cellcolor[RGB]{162,217,212}54 & \cellcolor[RGB]{137,187,237}63 & \cellcolor[RGB]{111,159,255}72 & \cellcolor[RGB]{111,138,255}81\\
        \hline
    \end{tabular}
\end{center}

九九乘法表口诀(背诵):

\begin{center}
\begin{tabular}{ p{5em} p{5em} p{5em} p{5em} p{5em} p{5em} p{5em} p{5em} p{5em}}
    \texttt{一一得一} & \texttt{一二得二} & \texttt{一三得三} & \texttt{一四得四} & \texttt{一五得五}\\
    \texttt{一六得六} & \texttt{一七得七} & \texttt{一八得八} & \texttt{一九得九} & \\
\end{tabular}
\end{center}

\begin{center}
    \begin{tabular}{ p{5em} p{5em} p{5em} p{5em} p{5em} p{5em} p{5em} p{5em} p{5em}}
        \texttt{二一得二} & \texttt{二二得四} & \texttt{二三得六} & \texttt{二四得八} & \texttt{二五一十}\\
        \texttt{二六一十二} & \texttt{二七一十四} & \texttt{二八一十六} & \texttt{二九一十八} & \\
        \texttt{三一得三} & \texttt{三二得六} & \texttt{三三得九} & \texttt{三四一十二} & \texttt{三五一十五}\\
        \texttt{三六一十八} & \texttt{三七二十一} & \texttt{三八二十四} & \texttt{三九二十七} & \\
        \texttt{四一得四} & \texttt{四二得八} & \texttt{四三一十二} & \texttt{四四一十六} & \texttt{四五二十}\\
        \texttt{四六二十四} & \texttt{四七二十八} & \texttt{四八三十二} & \texttt{四九三十六} & \\
        \texttt{五一得五} & \texttt{五二一十} & \texttt{五三一十五} & \texttt{五四二十} & \texttt{五五二十五}\\
        \texttt{五六三十} & \texttt{五七三十五} & \texttt{五八四十} & \texttt{五九四十五} & \\
        \texttt{六一得六} & \texttt{六二一十二} & \texttt{六三一十八} & \texttt{六四二十四} & \texttt{六五三十}\\
        \texttt{六六三十六} & \texttt{六七四十二} & \texttt{六八四十八} & \texttt{六九五十四} & \\
        \texttt{七一得七} & \texttt{七二一十四} & \texttt{七三二十一} & \texttt{七四二十八} & \texttt{七五三十五}\\
        \texttt{七六四十二} & \texttt{七七四十九} & \texttt{七八五十六} & \texttt{七九六十三} & \\
        \texttt{八一得八} & \texttt{八二一十六} & \texttt{八三二十四} & \texttt{八四三十二} & \texttt{八五四十}\\
        \texttt{八六四十八} & \texttt{八七五十六} & \texttt{八八六十四} & \texttt{八九七十二} & \\
        \texttt{九一得九} & \texttt{九二一十八} & \texttt{九三二十七} & \texttt{九四三十六} & \texttt{九五四十五}\\
        \texttt{九六五十四} & \texttt{九七六十三} & \texttt{九八七十二} & \texttt{九九八十一} & \\
    \end{tabular}
\end{center}

\subsubsection{多位数乘个位数}

使用竖式计算。

\begin{itemize}
\item 被乘数在上,乘数在下,个位对齐。
\item 从个位开始,从低到高计算。
\item 用乘数乘被乘数的个位,得到的结果写在加联上起第一行,个位与被乘数的个位对齐。
\item 用乘数乘被乘数的十位,得到的结果写在加联上起第二行,个位与被乘数的十位对齐。
\item 依此类推,用乘数乘被乘数的某位,得到的结果写在加联往下一行,个位与被乘数的这位对齐。
\item 为了防止数位对不齐,可以用〇补在结果后面的位上(并非必要)。
\item 将加联里所有的结果加起来,得到乘法的结果。
\end{itemize}

\subsubsection{多位数乘多位数}

使用竖式计算。

\begin{itemize}
\item 被乘数在上,乘数在下,个位对齐。
\item 从个位开始,从低到高计算。
\item 用乘数的个位乘被乘数,得到的结果写在加联上起第一行,个位与被乘数的个位对齐。
\item 用乘数的十位乘被乘数,得到的结果写在加联上起第二行,个位与被乘数的十位对齐。
\item 依此类推,用乘数的某位乘被乘数,得到的结果写在加联往下一行,个位与被乘数的这位对齐。
\item 为了防止数位对不齐,可以用〇补在结果后面的位上(并非必要)。
\item 将加联里所有的结果加起来,得到乘法的结果。
\end{itemize}

\subsection{自然数除法的计算}

\subsubsection{带余除法}

\begin{itemize}
\item 令余数等于被除数。
\item 如果余数小于除数,那么带余除法的商等于〇,带余除法的余数就是被除数。
\item 如果余数不小于除数,那么从余数中减去除数,同时商加一。
\item 不断重复,直到余数小于除数,这时的商就是带余除法的商,这时的余数就是带余除法的余数。
\end{itemize}

带余除法:

被除数 ÷ 除数 = 商 ··· 余数

被除数 = 除数 × 商 + 余数

\( 7 \div 3 = 2 \cdots 1 \),\( 3 \)除\( 7 \)得\( 2 \)余\( 1 \)。

余数总是小于除数的自然数。如果余数为〇,就说除数整除被除数。例:\( 3 \)整除\( 6 \)。所有不是〇的数都整除〇。

\subsubsection{竖式计算除法}

\begin{itemize}
\item 被除数写在竖式除号里,除数写在左外侧,商写在上方一行(下称商行)。
\item 计算从高位到低位。
\item 除数有几位,就看被除数的最高几位,作为除头。其余更低位的数称为除尾。除头应该不小于除数。如果被除数的最高几位比除数小,就多看一位。如果位数不够,就在被除数个位右边标上小数点,然后在右边逐位补〇,补到位数足够为止。
\item 用1到9作为试商乘以除数,得到试积。其中不大于除头的最大试积,对应的试商就是商的第一位,将它写到商行,与除头的最低一位对齐。
\item 将试积写在除头下方,最低位对齐。用除头减去试积,得到余数。余数写在试积下方,最低位和除头、试积对齐。
\item 如果余数为〇,且已经没有除尾(或者除尾也都是〇),说明除尽。把商行的数点上小数点(和被除数的小数点对齐),在小数点前的空位补上〇,就得到商。否则从除尾的最高位开始,从高到低,用除尾的数把上一次的余数补成新的除头(除头应该不小于除数)。
\item 继续用1到9作为试商,找出不大于除头的最大试积。将对应的试商写到商行,与除头的最低一位对齐。如果试商和上次的试商中间有空的位数,就补上〇。
\item 将试积写在除头下方,最低位对齐。用除头减去试积,再次得到余数。依此循环操作。
\item 如果某次除头和之前完全相同,且两次的除头都是被除数补〇得到的,说明除不尽,结果是循环小数。这时停止计算,不需要把这次的试商加入商行。这时,从第一次除头对应的试商到最后一个试商称为循环节。
\end{itemize}

\section{数论}

\subsection{倍数和因数}

如果甲数整除乙数,那么乙数是甲数的倍数,甲数是乙数的因数。例:\( 3 \)整除\( 6 \)。\( 6 \)是\( 3 \)的倍数,\( 3 \)是\( 6 \)的因数。

如果两个数的差是某个数的倍数,就说这两个数模这个数同余。例:\( 18 \)和\( 3 \)的差是\( 5 \)的倍数,就说\( 18 \)和\( 3 \)模\( 5 \)同余。

两个数共同的因数称为公因数,其中最大的称为最大公因数;共同的倍数称为公倍数,其中最小的称为最小公倍数。

两个数最大公因数是\( 1 \),就说两个数互素。

\subsection{素数与合数}

如果一个自然数的因数只有\( 1 \)和自己,就说它是素数。否则说它是合数。约定\( 0 \)即不是素数也不是合数。

\section{乘方}

简单介绍乘方:连乘的结果叫做乘方。

乘法可以更方便地表示若干个相同的数相加。比如,我们用\( 3 \times 4 \)表示\( 3+3+3+3 \)。
那么,能不能方便地表示若干个相同的数相乘呢?

我们把\( 3\times 3 \)称为\( 3 \)乘\( 2 \)次方,把\( 7\times 7\times 7\times 7\times 7 \)称为\( 7 \)乘\( 5 \)次方。

同一个数连乘几次,叫做它乘几次方。连乘的结果,叫做它的几次方或几次幂。这种运算叫做乘方或乘幂。

我们把\( 7 \)的\( 5 \)次方记作\( 7^5 \),把\( 7 \)称为底数,把\( 5 \)称为指数。
这样记法,比\( 7\times 7\times 7\times 7\times 7 \)更方便。

一个数的\( 1 \)次方就是它自己。一个数的\( 2 \)次方也叫做它的平方。一个数的\( 3 \)次方也叫做它的立方。

约定任何数的\( 0 \)次方是\( 1 \)。

\section{分数}

自然数除法的结果,可以直接用分数表示。分数号是高度居中的横杠。把被除数写在分数号上方,称为分子;把除数写在分数号下方,称为分母,就得到分数。

\[ 7 \div 3 = \frac{7}{3}\]

\subsection{倒数}

把分数的分子和分母对换位置,就得到它的倒数。〇没有倒数。

\subsection{约分}

如果分数的分子和分母都是某个数的倍数,可以将分子分母同除以这个数,得到分子分母都更小的分数。这个操作叫约分。

\subsection{分数的表示方法}

\begin{itemize}
\item 真分数和假分数:分子小于分母,就是真分数。分子大于等于分母,就是假分数。
\item 带分数:把假分数写成自然数与真分数的和。如果自然数和真分数都不是〇,就把两者放到一起,自然数在前,真分数在后,称为带分数。自然数称为带分数的整部,真分数称为带分数的余部。
\item 可以用带余除法把假分数转为带分数。
\item 把分子(被除数)转写为分母(除数)乘商加余数的和,则商就是整部,余数就是余部的分子,除数仍是分母。
\item 如果整除,余数为〇,则带分数变为自然数。
\end{itemize}

\subsection{分数的四则运算}

\subsubsection{分数的加减法}

使用通分来计算分数的加减法。

\begin{itemize}
\item 用被加数的分母乘以加数的分母,得到和的分母。
\item 用被加数的分子乘以加数的分母,加上被加数的分母乘以加数的分子,得到和的分子。
\end{itemize}

\begin{itemize}
\item 用被减数的分母乘以减数的分母,得到差的分母。
\item 用被减数的分子乘以减数的分母,减去被减数的分母乘以减数的分子,得到差的分子。
\end{itemize}

\subsubsection{分数的乘法}

乘以一个分数,就是除以分母然后乘以分子。

把被乘数和乘数看作分数。分子乘分子作为分子,分母乘分母作为分母。结果约分。

\subsubsection{分数的除法}

\begin{itemize}
\item 除以几,就是乘以几分之一。
\item 除以几分之一,就是乘以几。
\item 除以一个分数,就是乘以它的倒数。
\end{itemize}

除以一个数,就是乘以它的倒数。

比如,\( 3 \)除以\( 8 \),可以看作\( 3 \)个\( 1 \)除以\( 8 \),因此,又可以看作把\( 1 \)除以\( 8 \)的结果重复\( 3 \)次,也就是说:\( 3\div 8 = 3\times \frac{1}{8} \)。

\( 3 \)除以\( \frac{1}{8} \),就是\( 3\times 8 \)。

\( 3\div 8 = \frac{3}{8} \),因此\( \frac{3}{8} = 3\times \frac{1}{8} \)。因此,\( 4\div \frac{3}{8} = 4\div \left(3 \times \frac{1}{8}\right) = 4\div 3 \div \frac{1}{8} = 4\div 3 \times 8 = 4\times 8 \div 3 = 4\times \frac{8}{3} \)。

\subsection{用小数表示分数}

分数可以用小数的方式表示。小数点跟在个位右边,小数点左边是整部,小数点右边是小部。

通过竖式除法,可以得到分数的小数表示。
竖式除法中,如果通过补〇除尽,就得到有限小数,如果除不尽,就得到循环小数。

十进制位的名字:十分位、百分位、千分位等。和十、百、千位的顺序相反。有限小数只有有限位,或者说之后的位都是〇。

小数位数:从第几位的下一位开始都是〇的小数,就叫几位小数。

分数的整部就是带分数的整部,小部就等于带分数的余部。

小数点左移一位,数值是原来的十分之一;小数点右移一位,数值是原来的十倍。

小数比较大小的方法和自然数一样。

\subsubsection{小数转分数}

有限小数是整数除以\( 10 \)、\( 100 \)、\( 1000 \)等得到的结果。几位小数乘以\( 10 \)的几次方,就是整数。
因此有限小数可以转为分母为\( 10 \)的乘方的分数,然后约分。

循环小数的循环节,是整数除以\( 10-1 \)、\( 100-1 \)、\( 1000-1 \)等得到的结果。几位循环节的小数,用\( 10 \)的几次方减一乘,就是整数。

\subsubsection{小数的加减法}

竖式计算:

\begin{itemize}
\item 把被加数(被减数)和加数(减数)按小数点对齐,小数位数较少的,补〇。
\item 从最低位开始做加法(减法)。
\item 给结果点上小数点(和加减数对齐)。
\end{itemize}

\subsubsection{小数的乘法}

最好换成分数相乘,再换回小数。

竖式计算:

\begin{itemize}
\item 把被乘数和乘数的最低位对齐。
\item 计算乘法。
\item 被乘数的小数位数加上乘数的小数位数,就是结果的小数位数。按位数点上小数点,得到结果。
\end{itemize}

\subsubsection{小数的除法}

可以换成分数相除,再换回小数。

竖式计算:

\begin{itemize}
\item 如果除数是小数,就把被除数和除数的小数点同时向右移动相同的位数,直到除数变成整数。然后开始列竖式计算。
\item 和整数的竖式除法相比,要注意的是:被除数可能有小数部分。如果被除数有小数部分,就直接在后面接着补〇。
\item 计算的结果就是真正的结果。不需要把小数点移回去了。
\end{itemize}


\section{平面图形}

\subsection{直线和角}

\begin{itemize}
\item 线段:从一点到另一点可以画一条直直的线,叫做线段。这两点叫做线段的端点或顶点。
\item 射线:从线段一端出发,往另一端无限笔直延伸,得到射线。出发点叫射线的顶点。
\item 直线:从线段两端出发,分别往另一段无限笔直延伸,得到直线。
\item 补线:一个点把直线分成两条射线。这两条射线互为补线。
\item 距离:两点的距离是以它们为端点的线段的长度。
\item 角:角是一个顶点,以及从它发出的两条射线。这两条射线叫做角的边。
\item 始边和终边:从角的一条边出发,逆时针旋转,如果还没到达补线就到达角的另一条边,就说这个角是凸角,规定出发边叫始边,到达边叫做终边。否则说这个角叫凹角,规定出发边叫终边,到达边叫始边。这样,从始边出发,逆时针旋转,不会在终边前到达补线。
\item 角的方向:从一条边出发,可以顺时针转到另一条边,也可以逆时针转到另一条边。规定逆时针是正方向,顺时针是负方向。从一条边出发,逆时针转到另一条边,这样形成一个正角。从一条边出发,顺时针转到另一条边,这样形成一个负角。始边到终边,总形成正角。
\item 角的大小:把两个角的顶点对齐,再把始边对齐。如果终边也对齐,就说两个角一样大。如果从一个角的终边出发,可以顺时针转到另一个角的终边,就说这个角比另一个角大;否则就说它比另一个角小。
\item 平角:一个点把直线分成两条射线,叫做平角。平角由同一顶点出发的两条互补线构成。
\item 补角:从平角的顶点再画一条射线,把平角分成两个角。这两个角互为补角。
\item 直角:如果两个补角一样大,就说它们是直角。
\item 锐角和钝角:如果一个角比它的补角大,就说它是钝角,说它的补角是锐角。
\item 角度:我们把平角从始边到终边\( 180 \)等分,每份称为\( 1 \)度。平角一共有\( 180 \)度。直角有\( 90 \)度。钝角介于\( 90 \)和\( 180 \)度之间,锐角介于\( 0 \)和\( 90 \)度之间。
\item 〇角:始边和终边重合的角叫做〇角。一条射线可以理解为〇角。
\item 角的加减法:两个角的顶点对齐后,把一个角的终边和另一个角的始边对齐,那么这个角的始边到另一个角的终边就叫做两个角的和角。把两个角的终边对齐,两角的始边就形成两个角的差角。两个角一样大,差角就是〇角。否则较大角的始边到较小角的始边是正角。
\end{itemize}

\begin{itemize}
\item 平行:两条直线不相交,就说它们平行。两条线段、射线所在的直线不相交,就说它们平行。
\item 交角:两条直线相交,以交点为顶点,两条直线的某条边为边,形成的角,叫做两条直线的交角。
\item 垂直:交角是直角,就说两条直线垂直。
\item 垂线:过直线外一点恰好可以画一条到它的垂线。垂线和直线交于一点,叫做垂足。点到垂足的线段长度叫做点到直线的距离。
\end{itemize}

\subsection{多边形}

线段首尾相连,得到多边形。线段称为多边形的边;线段的端点称为多边形的顶点。

\subsubsection{三角形}
三个线段首尾相连,得到三角形。三角形有三条边,三个顶点。每个顶点属于两条边,称为它的邻边;第三条边称为它的对边。每个顶点和两条邻边所在射线构成的角,从始边到终边,称为三角形关于这个顶点的内角。内角的补角称为三角形关于这个顶点的外角。三角形有三个内角。

如果三角形有一个内角是直角,就说它是直角三角形。如果有一个内角是钝角,就说它是钝角三角形。如果三个角都是锐角,就说它是锐角三角形。

如果三角形的三条边中有两条边等长,就说它是等腰三角形,等长的两条边称为它的腰。如果三边都等长,就说它是正三角形。

过三角形的顶点到它对边的距离称为三角形关于对边的高,这时对边称为底边。

\subsection{四边形}
四个线段首尾相连,得到四边形。四边形有四条边,四个顶点。每个顶点属于两条边,称为它的邻边。每条边与两个顶点不相连,这两个顶点是另一条边的端点,称为这条边的对边。

如果每条边都和对边平行,就说四边形是平行四边形。平行四边形的对边总等长。

如果平行四边形顶点的邻边形成的正角总是直角,就说它是矩形或方形。矩形的两对边里,如果有一边比另一边长,就说它是长方形。较长边叫做长方形的长,较短边是长方形的宽。

如果四边形的四条边等长,就说它是菱形。

如果矩形的四条边等长,就说它是正方形。

\subsection{圆}

到一点距离相等的所有点组成的图形,称为圆。这点称为圆心,圆心到圆上一点的距离称为圆的半径。过圆心的直线总和圆交于两点,称为对径点。圆上一点和对径点的距离称为圆的直径。半径是直径的一半。

\subsection{周长和面积}

多边形的周长是全部边长之和。

多边形的面积是它的边围成的部分的面积。

正方形的面积是边长乘边长。长方形的面积等于长乘以宽。平行四边形的高,指过一个顶点到对边的垂线的长度,对应的两条对边称为底边。平行四边形的面积等于底边长乘以高。三角形的面积是同底边等高的平行四边形的一半。

\end{document}